\section{Напряжение и термодинамическая корректность}

В качестве меры напряжения образца мы используем \textbf{второй тензор напряжений Пиолы-Кирхгофа}, который вычисляется по цепному правилу дифференцированием энергии \(\psi\) 
по правому тензору деформации Коши-Грина \(\vect C\):

\begin{equation}
  \mathbf{S} \;=\; 2\,\frac{\partial \psi}{\partial \mathbf{C}}
  \;=\; 2\,\frac{\partial \psi}{\partial \boldsymbol\xi} \cdot \frac{\partial \boldsymbol\xi}{\partial \mathbf{C}}
  \;=\; 2\,\mathbf{r}(\boldsymbol\xi)\cdot\frac{\partial \boldsymbol\xi}{\partial \mathbf{C}},
  \qquad \mathbf{r}:=\frac{\partial \psi}{\partial \boldsymbol\xi}.
  \label{eq:chain-rule}
\end{equation}
Вектор $\frac{\partial \boldsymbol\xi}{\partial \mathbf{C}}$ называют базисом, он является известным \colorbox{yellow}{В смысле расхожим, или известным из экспериментальных данных?} аналитическим выражением, зависящим от выбранной меры деформации; $\vect{r} = (r_1, r_2, r_3)$ -- функцией отклика, которая является итоговым \textit{искомым} в процессе обучения \textit{data-driven} определяющего соотношения.   

Такое построение имеет ключевые следствия:
\newline
\textit{Объективность:} $\psi(\mathbf{C})=\psi(\mathbf{Q}^\top\mathbf{C}\mathbf{Q})$ для любой ортогональной $\mathbf{Q}$, а значит и $\mathbf{S}$ инвариантен к поворотам.
\newline
\textit{Симметрия напряжений:} $\mathbf{S}=\mathbf{S}^\top$ вследствие симметрии $\mathbf{C}$ и корректного применения цепного правила.
\newline
\textit{Термодинамическая корректность:} равенство \eqref{eq:chain-rule} является следствием неравенства Клаузиуса-Дюгема 
$\mathcal{D} = \mathbf{S} : \dot{\mathbf{C}} - \dot{\psi}(\mathbf{C}) \geq 0$, 
выражающее второе начало термодинамики для механических процессов \cite{truesdell1984historical,truesdell2004nonlinear}.
% \begin{itemize}
%   \item \textbf{Объективность:} $\psi(\mathbf{C})=\psi(\mathbf{Q}^\top\mathbf{C}\mathbf{Q})$ для любой ортогональной $\mathbf{Q}$, а значит и $\mathbf{S}$ инвариантен к поворотам.
%   \item \textbf{Симметрия напряжений:} $\mathbf{S}=\mathbf{S}^\top$ вследствие симметрии $\mathbf{C}$ и корректного применения цепного правила.
%   \item \textbf{Термодинамическая корректность:} равенство \eqref{eq:chain-rule} является следствием неравенства Клаузиуса-Дюгема 
%   $\mathcal{D} = \mathbf{S} : \dot{\mathbf{C}} - \dot{\psi}(\mathbf{C}) \geq 0$, 
%   выражающее второе начало термодинамики для механических процессов \cite{truesdell1984historical,truesdell2004nonlinear}.
% \end{itemize}

% \textbf{Связь тензора Лапласа и второго тензора напряжений Пиолы-Кирхгофа}

 % Необходимо установить связь деформации Лапласа $\vect{\xie}$ и второго тензора напряжений Пиолы-Кирхгофа $\vect{S}$. Применяя цепное правило дифференцирования к выражению \eqref{eq:chain-rule} и используя меру деформации Лапласа,

Расписав покомпонентно уравнение \eqref{eq:chain-rule} и подставив туда известный базис $\frac{\partial \boldsymbol\xi}{\partial \mathbf{C}}$, получаем аналитические выражения для компонент второго тензора напряжений Пиолы-Кирхгофа в двумерном случае:

\begin{equation}
\begin{aligned}
  S_{11} &= e^{-2\xi_1}\big(r_1-2\xi_3 r_3\big) + e^{-2\xi_2} r_2\,\xi_3^2,\\
  S_{22} &= e^{-2\xi_2} r_2,\\
  S_{12} &= -e^{-2\xi_2} r_2\,\xi_3 + e^{-2\xi_1} r_3,
\end{aligned}
\label{eq:stress_components_2d}
\end{equation}
% $r_1, r_2, r_3$ — компоненты функции отклика $\mathbf{r} = \frac{\partial \psi}{\partial \boldsymbol\xi}$.

\textbf{Ограничения на гиперупругую модель\colorbox{yellow}{чего? на что? Лучше первым абзацем архитектуры, как требования к гиперупругому потенциалу}}

В соответствии с принципами термодинамики и механики сплошных сред, 
гиперупругая модель $\psi$ должна удовлетворять ряду фундаментальных ограничений, обеспечивающих физическую корректность и 
материальную устойчивость:
\newline
\textit{Неотрицательность} исключает отрицательную энергию деформации
  \begin{equation}
    \psi(\vect{\xi}) \ge 0\quad \forall\,\vect{\xi}\in\mathbb{R}^3.
    \label{eq:non_negativity}
  \end{equation}
  % Это 
\newline  
 \textit{Нулевые значения для $\psi$ и $\vect S$ в естественном состоянии} означают, что недеформированное (начальное) состояние среды не имеет остаточных напряжений
  \begin{equation}
    \psi(\vect 0)=0,\qquad \vect S(\vect I)=\vect 0,
    \label{eq:natural_state_stress}
  \end{equation}
\newline
  \textit{Бесконечный рост (коэрцитивность)}. Неограниченный рост меры деформации физически недостижим при конечной работе \colorbox{yellow}{не совсем понятно, уместно ли говорить про $J\to\infty$ для случая несжимаемой мембраны}
  \begin{equation}
    \psi(\vect{\xi}) \to \infty\ \text{при}\ \lVert\vect{\xi}\rVert\to\infty,\qquad
    \vect{S} \to \infty\ \text{при}\ J\to\infty \text{ или } J\to 0^{+},\qquad
    J=\det\vect F,
    \label{eq:energy_constraints}
  \end{equation}
  % Это обеспечивает коэрцивность: крайние объёмные деформации ($J\to\infty$, $J\to 0^{+}$) и неограниченный рост меры деформации физически недостижимы при конечной работе.

Свойства \eqref{eq:non_negativity} -- \eqref{eq:energy_constraints} принято записывать через градиент деформации \(\vect{F}\) и правый тензор деформации Коши-Грина \(\vect{C}\) 
\cite{antman2005nonlin,green1839laws,kirchhoff1850gleichgewicht}, но они эквивалентны и для меры деформации Лапласа \(\vect{\xi}\).

% Свойства \eqref{eq:non_negativity} -- \eqref{eq:energy_constraints} 


