\section{Напряжение и термодинамическая корректность}

В качестве меры напряжения образца мы используем \textbf{второй тензор напряжений Пиолы-Кирхгофа}, который вычисляется по цепному правилу дифференцированием энергии \(\psi\) 
по правому тензору деформации Коши-Грина \(\mathbb C\):

\begin{equation}
  \mathbb{S} \;=\; 2\,\frac{\partial \psi}{\partial \mathbb{C}}
  \;=\; 2\,\frac{\partial \psi}{\partial \vect{\xi}} \cdot \frac{\partial \vect{\xi}}{\partial \mathbb{C}}
  \;=\; 2\,\vect{r}(\vect{\xi})\cdot\frac{\partial \vect{\xi}}{\partial \mathbb{C}},
  \qquad \vect{r}:=\frac{\partial \psi}{\partial \vect{\xi}}.
  \label{eq:chain-rule}
\end{equation}
Тензор $\frac{\partial \boldsymbol\xi}{\partial \mathbf{C}}$ называют базисом, он является известным аналитическим выражением, зависящим от выбранной меры деформации; $\vect{r} = (r_1, r_2, r_3)$ — функция отклика, которая выучивается из данных в процессе обучения data-driven модели.

Записывая уравнение \eqref{eq:chain-rule} покомпонентно и подставляя известный базис $\frac{\partial \vect{\xi}}{\partial \mathbb{C}}$, получаем аналитические выражения для компонент второго тензора напряжений Пиолы-Кирхгофа в двумерном случае:
\begin{equation}
\begin{aligned}
  S_{11} &= e^{-2\xi_1}\big(r_1-2\xi_3 r_3\big) + e^{-2\xi_2} r_2\,\xi_3^2,\\
  S_{22} &= e^{-2\xi_2} r_2,\\
  S_{12} &= -e^{-2\xi_2} r_2\,\xi_3 + e^{-2\xi_1} r_3.
\end{aligned}
\label{eq:stress_components_2d}
\end{equation}
Здесь $S_{11}$, $S_{22}$ и $S_{12}$ — компоненты в материальном базисе $\vect{G}^{\alpha}\otimes\vect{G}^{\beta}$, согласованные с представлением $\tilde{\mathbb{F}}_{2d}$ в уравнениях \eqref{eq:xi-F}--\eqref{eq:F-C}.

Такое построение имеет ключевые следствия:
\newline
\textit{Объективность:} $\psi(\mathbb{C})=\psi(\mathbb{Q}^\top\mathbb{C}\mathbb{Q})$ для любой ортогональной $\mathbb{Q}$, а значит и $\mathbb{S}$ инвариантен к поворотам.
\newline
\textit{Симметрия напряжений:} $\mathbb{S}=\mathbb{S}^\top$ вследствие симметрии $\mathbb{C}$ и корректного применения цепного правила.
\newline
\textit{Термодинамическая корректность:} равенство \eqref{eq:chain-rule} является следствием неравенства Клаузиуса-Дюгема 
$\mathcal{D} = \mathbb{S} : \dot{\mathbb{C}} - \dot{\psi}(\mathbb{C}) \geq 0$, 
выражающее второе начало термодинамики для механических процессов \cite{truesdell1984historical,truesdell2004nonlinear}.

В соответствии с принципами термодинамики и механики сплошных сред, 
гиперупругая модель $\psi$ должна удовлетворять ряду фундаментальных ограничений, обеспечивающих физическую корректность и 
материальную устойчивость:
\newline
\textit{Неотрицательность} исключает отрицательную энергию деформации
  \begin{equation}
    \psi(\vect{\xi}) \ge 0\quad \forall\,\vect{\xi}\in\mathbb{R}^3.
    \label{eq:non_negativity}
  \end{equation}
\newline  
 \textit{Нулевые значения для $\psi$ и $\mathbb S$ в естественном состоянии} означают, что недеформированное (начальное) состояние среды не имеет остаточных напряжений
  \begin{equation}
    \psi(\vect 0)=0,\qquad \mathbb S(\mathbb I)=\vect 0,
    \label{eq:natural_state_stress}
  \end{equation}
\newline
  \textit{Бесконечный рост (коэрцитивность)}. 
  \begin{equation}
    \psi(\vect{\xi}) \to \infty\ \text{при}\ \lVert\vect{\xi}\rVert\to\infty,\qquad
    \mathbb{S} \to \infty\ \text{при}\ J\to\infty \text{ или } J\to 0^{+},\qquad
    J=\det\mathbb F,
    \label{eq:energy_constraints}
  \end{equation}

Свойства \eqref{eq:non_negativity} -- \eqref{eq:energy_constraints} принято записывать через градиент деформации \(\mathbb{F}\) и правый тензор деформации Коши-Грина \(\mathbb{C}\) 
\cite{antman2005nonlin,green1839laws,kirchhoff1850gleichgewicht}, но они эквивалентны и для меры деформации Лапласа \(\vect{\xi}\).
