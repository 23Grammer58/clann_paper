\section{Кинематика}
\textbf{Основные соотношения}

Мы рассматриваем равновесие тонкой несжимаемой гиперупругой мембраны с толщиной $H$ под
определенными нагрузками.
Деформация мембраны характеризуется деформацией её срединной поверхности. 
Обозначим через \(\mathbf{X}\) и \(\mathbf{x}\) положения точек, 
в соответствующих базисах \(\vect{E}_{\alpha}\) и \(\vect{e}_{\alpha}\), 
в исходной (недеформированной) \(\Omega_0 \subset \mathbb{R}^2\) и текущей (деформированной) \(\Omega_t \subset \mathbb{R}^2\)
конфигурациях поверхности мембраны соответственно. 
Деформация определяется отображением \(\mathbf{x} = \mathbf{x}(\mathbf{X})\), 
поверхностный градиент деформации \(\mathbf{F} = \vect{e}_{\alpha} \otimes \vect{E}^{\alpha}\),
а правый тензор Коши—Грина \(\vect C = C_{\alpha\beta} \,\vect{e}_{\alpha} \otimes \vect{e}_{\beta} = \vect F^{\top} \vect F\). 
Для определения меры деформации мы используем меру Лапласа \(\vect{\xi} = (\xi_1 , \xi_2 , \xi_3)^T\) \cite{xi2023},
которая может быть вычислена двумя эквивалентными способами: 
либо через QR-разложение градиента деформации \(\vect F = \vect Q \vect R\) с \(\vect U = \vect R\) , 
либо через разложение Холецкого правого тензора Коши-Грина \(\vect C = \vect U^{\top}\vect U\) (Приложение \ref{app:cholesky}).
В этом случае гиперупругий потенциал является функцией от деформации Лапласа \(\psi = \psi(\vect{\xi})\).


\textbf{Мера деформации Лапласа}
В двумерном случае вводятся характеристики
\begin{equation}
\xi_1 = \ln(u_{11}),\quad \xi_2 = \ln(u_{22}),\quad \xi_3 = \frac{u_{12}}{u_{11}}, 
\quad \vect{U} = u_{\alpha\beta} \; \vect{e}_{\alpha} \otimes \vect{e}_{\beta}.
\label{eq:laplace_coords}
\end{equation}


