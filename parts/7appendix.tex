\appendix

\chapter{\texorpdfstring{Эквивалентность QR-факторизации $\vect F$ и разложения Холецкого $\vect C=\vect F^{\top}\vect F$ для вычисления логарифмических координат $\boldsymbol{\xi}$}{Эквивалентность QR и Холецкого}}
\label{app:cholesky}

\section{Постановка и обозначения}

Рассматривается двумерная гиперупругая кинематика. Пусть:
\begin{itemize}
  \item $\vect F \in \mathbb{R}^{2 \times 2}$ — градиент деформации, $\det \vect F > 0$,
  \item $\vect C = \vect F^{\top}\vect F$ — правый тензор Коши–Грина (симметричный положительно определённый, SPD),
  \item Холецкий: $\vect C = \tilde{\vect F}^{\top}\tilde{\vect F}$, где $\tilde{\vect F}$ — верхнетреугольная и $\text{diag}(\tilde{\vect F}) > 0$,
  \item Логарифмические координаты:
    $\boldsymbol{\xi} = (\xi_1, \xi_2, \xi_3) = (\ln u_{11}, \ln u_{22}, u_{12}/u_{11})$.
\end{itemize}

Цель: показать, что при наличии $\vect F$ можно заменить вычисление $\tilde{\vect F} = \text{chol}(\vect C)$ на $\tilde{\vect F} = \vect R$ из тонкого QR($\vect F$) = $\vect Q \vect R$ (с $\text{diag}(\vect R) > 0$), и получить те же $\boldsymbol{\xi}$.

\section{Теорема (эквивалентность U и R)}

Пусть $\vect F \in \mathbb{R}^{2 \times 2}$ невырождённая ($\det \vect F > 0$). Рассмотрим тонкую QR-факторизацию
\begin{equation}
\vect F = \vect Q \vect R,
\end{equation}
где $\vect Q \in \mathbb{R}^{2 \times 2}$ — ортогональная ($\vect Q^{\top}\vect Q = \vect I$), $\vect R \in \mathbb{R}^{2 \times 2}$ — верхнетреугольная. Выберем стандартную нормализацию $\text{diag}(\vect R) > 0$. Тогда $\vect R$ совпадает с фактором Холецкого для $\vect C$:
\begin{equation}
\vect R = \text{chol}(\vect C), \quad \text{с} \quad \vect C = \vect F^{\top}\vect F.
\end{equation}

\textbf{Доказательство.}
\begin{equation}
\vect C = \vect F^{\top}\vect F = (\vect Q \vect R)^{\top}(\vect Q \vect R) = \vect R^{\top} \vect Q^{\top} \vect Q \vect R = \vect R^{\top} \vect R.
\end{equation}
Так как $\vect C$ — SPD и $\vect R$ — верхнетреугольная с положительной диагональю, то представление $\vect C = \vect R^{\top}\vect R$ единственно. По единственности фактора Холецкого (с $\text{diag} > 0$) следует $\vect R = \text{chol}(\vect C)$. $\square$

\textbf{Следствие.} Логарифмические координаты $\boldsymbol{\xi}$, определённые через $\tilde{\vect F} = \text{chol}(\vect C)$, можно эквивалентно вычислять из $\tilde{\vect F} = \vect R$ в QR($\vect F$), при условии $\text{diag}(\vect R) > 0$.

\section{Координаты $\vect{\xi}$ через $\tilde{\vect F}$}

Для $\tilde{\vect F} = \begin{bmatrix} \tilde f_{11} & \tilde f_{12} \\ 0 & \tilde f_{22} \end{bmatrix}$, $\text{diag}(\tilde{\vect F}) > 0$,
\begin{equation}
\boldsymbol{\xi} = (\xi_1, \xi_2, \xi_3) = (\ln \tilde f_{11}, \ln \tilde f_{22}, \tilde f_{12}/\tilde f_{11}).
\end{equation}
Тем самым, $\boldsymbol{\xi}(\vect F) := \boldsymbol{\xi}(\vect R(\vect F)) = \boldsymbol{\xi}(\tilde{\vect F}(\vect C))$.


