% ===== CLaNN: однослойная ICNN (наглядная архитектура) =====
\begin{tikzpicture}[
  font=\small, node distance=9mm and 14mm, >={Latex},
  box/.style={draw, rounded corners, align=center, minimum height=7mm, inner sep=2mm},
  block/.style={box, fill=green!6, text width=44mm},
  pin/.style={circle, draw, minimum size=6mm, fill=white},
  neur0/.style={circle, draw, minimum size=6mm, fill=white},
  neur/.style={circle, draw, minimum size=6mm, fill=white},
  title/.style={font=\bfseries},
  every fit/.style={draw, rounded corners, inner sep=3mm}
]

% Входы (лог‑лапласовы координаты)
\node[pin] (x1) {$\xi_1$};
\node[pin, below=6mm of x1] (x2) {$\xi_2$};
\node[pin, below=6mm of x2] (x3) {$\xi_3$};

% Аффинная проекция
\node[neur0, right=14mm of x1, yshift=12mm] (h11) {};
\node[neur0, right=14mm of x1] (h12) {};
\node[neur0, right=14mm of x1, yshift=-12mm] (h13) {};
\node[align=center, right=14mm of x1, yshift=-22mm] (vdots0) {$\vdots$};
\node[neur0, right=14mm of x1, yshift=-34mm] (h14) {};
\draw[->] (x1) -- (h11.west);
\draw[->] (x2) -- (h11.west);
\draw[->] (x3) -- (h11.west);
\draw[->] (x1) -- (h12.west);
\draw[->] (x2) -- (h12.west);
\draw[->] (x3) -- (h12.west);
\draw[->] (x1) -- (h13.west);
\draw[->] (x2) -- (h13.west);
\draw[->] (x3) -- (h13.west);
\draw[->] (x1) -- (h14.west);
\draw[->] (x2) -- (h14.west);
\draw[->] (x3) -- (h14.west);
% \node[block, above=2mm of h11, text width=36mm] (aff) {$s = W_1\,\xi + b_1$};
\node[block, above=2mm of h11, text width=60mm, inner sep=3mm] (soft) {$z_i = \mathrm{softplus}(\beta (W_1\,\xi + b_1))/\beta$};

% % Скрытый слой (softplus): три нейрона, троеточие и ещё один
% \node[neur, right=14mm of h12, yshift=12mm] (h1) {};
% \node[neur, right=14mm of h12] (h2) {};
% \node[neur, right=14mm of h12, yshift=-12mm] (h3) {};
% \node[align=center, right=14mm of h12, yshift=-22mm] (vdots1) {$\vdots$};
% \node[neur, right=14mm of h12, yshift=-34mm] (h4) {};
% \draw[->] (h11.east) -- (h1.west);
% \draw[->] (h12.east) -- (h2.west);
% \draw[->] (h13.east) -- (h3.west);
% \draw[->] (h14.east) -- (h4.west);
% % Блок активации над нейронами
% \node[block, above=2mm of h1, text width=36mm] (soft) {$z_i = \mathrm{softplus}(\beta (W_1\,\xi + b_1)/\beta$};

% Линейный выход (неотрицательные веса)
\node[block, right=22mm of h12, text width=42mm] (readout) {$\psi(\boldsymbol{\xi}) = \mathbf{W}_2^{\top}\big(z - z_0\big) - \mathbf{r}_0^{\top}\boldsymbol{\xi}$};
\draw[->] (h11.east) -- (readout.west);
\draw[->] (h12.east) -- (readout.west);
\draw[->] (h13.east) -- (readout.west);
\draw[->] (h14.east) -- (readout.west);

% % Центрирование \to физическая энергия
% \node[block, below=10mm of readout] (psiphys) {$\psi_{\mathrm{phys}}(\boldsymbol{\xi}) = \psi(\boldsymbol{\xi}) - \mathbf{r}_0^{\top}\boldsymbol{\xi}$ \\
%   \scriptsize $r_0 = W_1^{\top}\!(w_2^{+}\!\odot\!\sigma(\beta b_1))$};
% \draw[->] (readout.south) -- (psiphys.north);

% Рамка и заголовок
\node[fit=(x1)(x3)(soft)(readout), label={[title]above:{Архитектура CLaNN: однослойная ICNN}}] (archfit) {};

\end{tikzpicture}