\section{Заключение}

В работе предложена физически информированная архитектура CLaNN для гиперупругих материалов, основанная на выпуклой
потенциальной энергии деформации и лог\textendash лапласовой параметризации кинематики.

Архитектура обеспечивает термодинамическую согласованность; благодаря
выпуклости задача решается как гладкая выпуклая минимизация с предсказуемой сходимостью градиентных и
квазиньютоновских методов. 
При этом в архитектуре для вычисления тензора напряжений явно не используется информация о сжимаемости/несжимаемости изотропности/анизотропности
материала, что позволяет использовать CLaNN для решения задач с различными типами материалов.

В тестах интерполяции CLaNN достигает малых ошибок при наличии репрезентативных обучающих данных; в режиме
экстраполяции сохраняет устойчивость и физическую правдоподобность отклика,
тогда как локально\textendash интерполяционные DD\textendash модели (k\textendash NN/IDW) проявляют артефакты за пределами окна обучения.

На численных экспериментах раздувания закреплённой круглой мембраны (гомогенная и гетерогенная толщина) CLaNN корректно
воспроизводит поля перемещений и напряжений и демонстрирует быстрое, предсказуемое сходимостное поведение в единой
КЭ\textendash постановке для всех сравниваемых моделей.

По вычислительной эффективности CLaNN превосходит kNN\textendash основанную DD\textendash модель: на гомогенной задаче выигрыш составляет
порядка $\times 1.9$ за счёт отсутствия дорогостоящих k\textendash NN/IDW\textendash запросов и внешних проекций на данные на каждой итерации. 
Стоит отметить, что полученная величина выигрыша может быть и выше за счет оптимизации взаимодействия решателя и CLaNN, а также подбора оптимальных гиперпараметров.
На гетерогенной толщине CLaNN сохраняет работоспособность без специальных эвристик, 
тогда как DD\textendash модель требует дополнительной регуляризации и/или интерполяции данных в окрестности малых деформаций.
Также из-за отсутсвия критерия остановки расчета по норме невязки у релаксационных методов решения задачи равновесия может привести к сложноценимому увеличению времени расчета
по сравнения с ньютоновскими методами, с которыми позволяет работать CLaNN.

В итоге CLaNN объединяет механическую корректность и эффективность нейросетей: выпуклая энергия и дифференцируемость обеспечивают устойчивое 
решение вариационных задач и ускоряют расчёт по сравнению с классическими DD\textendash подходами, а также высокую способность к аппроксимации гиперупругих материалов
на малых выборках данных. 
Дальнейшие направления: протестировать анизотропные материалы и реальные экспериментальные данные.


