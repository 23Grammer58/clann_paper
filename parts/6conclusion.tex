

\section{Дискуссия}

В данной работе предложена и реализована физически-информированная архитектура CLaNN для гиперупругих материалов, основанная на выпуклой потенциальной энергии деформации и меры деформации Лапласа.

В исследованиях по основанной на данных механике \cite{KirchdoerferOrtiz2016,salamatova2020hyperelastic} отклик строится как локальная интерполяция (Voronoi, k-NN) в пространствах деформаций-напряжений, что ведет к невыпуклой задаче равновесия и необходимости применения релаксационных стратегий решения, вместо градиентных и второпорядковых методов со строгими гарантиями сходимости (квазиньютоновские схемы, Ньютон и т.д.). Такие методы устойчивы, но, как правило, требуют существенно большего числа шагов нагружения и внутренних итераций (а также повторяющихся k-NN/IDW-запросов), что приводит к росту времени расчёта. 

Локальную интерполяцию мы заменили интерполяцией таблично-заданного в мерах Лапласа определяющего соотношения \cite{salamatova2020hyperelastic} выпуклым по входу нейропотенциалом на базе ICNN. Такой подход априори удовлетворяет требованию к объективности, материальной симметрии \cite{freed2020laplace}, симметрии тензора напряжения и выпуклости гиперупругой модели. Последнее достигалось архитектурно, через монотонно неубывающую функцию активации и неотрицательные веса в соответствии с \cite{icnn2017}. Для построения гиперупругой модели CLaNN мы провели виртуальный аналог двухосного экспериментального исследования (по протоколу из \cite{sommer2015biomechanical}) крестообразного образца неогуковского материала. Входными данными для обучения модели CLaNN стали пары деформация-напряжение $\boldsymbol{\xi}_r^{(n)}, \mathbb{S}_r^{(n)}$, извлеченные из виртуального эксперимента. Полученная модель продемонстрировала высокую точность интерполяции для главных компонент напряжений при растяжении обучающих данных.

В работе \cite{tacc2024benchmarking} авторы сравнивают подходы к моделированию гиперупругости физически-информированными архитектурами CANN (Constitutive Artificial Neural Networks), ICNN (Input Convex Neural Networks), NODE (Neural Ordinary Differential Equations). Для изотропных эластомеров (при одноосных, равнодвухосных, сдвиговых модах дефомирования) показано, что обучение физически дополненных сетей на равнодвухосном растяжении обеспечивает лучшую межрежимную экстраполяцию. Подобным образом мы протестировали экстраполяционную способность трех вариантов CLaNN, обученных на данных сложного двухосного протокола, извлеченных из разных областей образца. Моделировалось два инфляционных сценария -- раздувание гомогенной по толщине мембраны и гетерогенной мембраны с утолщениями-концентраторами, продуцирующими неоднородные поля деформаций со значительными сдвиговыми компонентами тензора напряжений в полюсе образца.
Точность наших решений, полученная с CLaNN, сравнивалась с референтным конечно-элементным решением для раздутия неогуковской модели мембраны с тем же параметром модуля сдвига, с которым генерировались двухосные данные для обучения CLaNN. При сопоставимой с неогуковским решением скоростью CLaNN корректно предсказывает поля перемещений и напряжений при инфляции гомогенной и гетерогенной по толщине круглой гиперупругой мембраны. Абсолютные и относительные интегральные ошибки норм перемещений и напряжений падают при увеличении области извлечения данных для обучения CLaNN -- к дужкам крестообразного образца растет порядок сдвиговой компоненты напряжений. Это соответствует наблюдениям из бенчмарка \cite{tacc2024benchmarking}: с расширением покрытия экспериментальными данными пространства деформаций при обучении CANN/ICNN/NODE, растет точность межрежимной экстраполяции. 

Отметим, что по вычислительной эффективности CLaNN превосходит методы локальной интерполяции: на гомогенной задаче выигрыш составляет
порядка $\times 141.8$ за счёт отсутствия дорогостоящих k\textendash NN/IDW\textendash запросов и внешних проекций на данные на каждой итерации. 
Стоит отметить, что полученная величина выигрыша может быть и выше за счет оптимизации взаимодействия решателя и CLaNN, а также подбора оптимальных гиперпараметров.
На гетерогенной толщине CLaNN сохраняет работоспособность без специальных эвристик, 
тогда как в предыдущей работе \cite{salamatova2020hyperelastic} таблично\textendash заданная гиперупругая модель требует дополнительной регуляризации и/или интерполяции данных в окрестности малых деформаций.
Также из-за отсутствия критерия остановки расчета по норме невязки у релаксационных методов решения задачи равновесия может привести к сложнооценимому увеличению времени расчета по сравнения с ньютоновскими методами, с которыми позволяет работать CLaNN.

В работе \cite{ddaniso2024} была показана применимость таблично-заданных определяющих соотношений в мерах Лапласа для анизотропных биоматериалов. При этом инварианты для описания анизотропии не вводились. Таблично-заданное определяющее соотношение строилось по синтетическим экспериментальным данным, которые были получены с гиперупругой моделью Хользапфеля-Гассера-Огдена \cite{ddaniso2024} для свиной кожи. Трех функций отклика, зависящих от трех соответствующих компонент тензора Лапласа было достаточно для описания механического поведения анизотропного материала в двумерной постановке. Это обнадеживает применение CLaNN для анизотропного материала без введения предположений о симметриях материала, как это недавно делалось родственных работах с ICNN определяющим соотношениям, базирующимся на инвариантах и псевдо-инвариантах правого тензора деформаций Коши-Грина \cite{kalina2025neural}. В будущем мы планируем обучить модель CLaNN на экспериментальных данных, полученных при тестировании перикардиальной ткани.
Как итог, на фоне феноменологических гиперупругих моделей CLaNN не требует предположений о форме потенциала и предлагает гибкость универсального аппроксиматора потенциала при удовлетворении законов термодинамики. На фоне локально интерполяционных подходов, основанных на данных CLaNN возвращает в постановку гладкость и выпуклость энергии, что превращает задачу равновесия в хорошо обусловленную минимизацию с предсказуемой сходимостью и выигрышем по времени.
Ограничения. Наши результаты демонстрируют, что мы можем успешно применять архитектуру CLaNN для моделирования гиперупругой изотропной мембраны. Однако мы столкнулись с некоторыми ограничениями, которые указывают на будущую работу. Во-первых, в данной работе предполагается гиперупругость мембраны. Во-вторых, мы рассматриваем только изотропный случай в двумерных сценариях. В дальнейшем мы расширим применение CLaNN на анизотропию, устранив это ограничение аппроксимацией таблично-заданного определяющего соотношения в мерах Лапласа для анизотропного материала.

In this work we propose and implement a physics‑informed CLaNN architecture for hyperelastic materials, based on a convex strain‑energy potential and the Laplace strain measure.

In data‑driven computational mechanics \cite{KirchdoerferOrtiz2016,salamatova2020hyperelastic}, the material response is often constructed via local interpolation (Voronoi, k‑NN) in strain–stress spaces. This leads to a nonconvex equilibrium problem and necessitates relaxation‑type solution strategies instead of gradient‑based and second‑order methods with strong convergence guarantees (quasi‑Newton schemes, Newton–Raphson, etc.). Such relaxation methods are robust but typically require many more load increments and internal iterations (as well as repeated k‑NN/IDW queries), which increases runtime.

We replace local interpolation of a tabulated constitutive relation in Laplace strain measures \cite{salamatova2020hyperelastic} with an input‑convex neural potential based on ICNN. This approach a priori satisfies objectivity, material symmetry \cite{freed2020laplace}, stress‑tensor symmetry, and convexity of the hyperelastic model. The latter is enforced architecturally via a monotonically nondecreasing activation function and nonnegative weights, in accordance with \cite{icnn2017}. To build the CLaNN hyperelastic model, we performed a virtual analog of a biaxial experimental study (following \cite{sommer2015biomechanical}) on a cruciform specimen of a Neo‑Hookean material. The training inputs for CLaNN were strain–stress pairs, $\boldsymbol{\xi}_r^{(n)}, \mathbb{S}_r^{(n)}$, extracted from the virtual experiment. The resulting model exhibited high interpolation accuracy for the principal stress components under the tensile states included in the training set.

In \cite{tacc2024benchmarking}, the authors compare physics‑informed architectures—CANN (Constitutive Artificial Neural Networks), ICNN (Input Convex Neural Networks), and NODE (Neural Ordinary Differential Equations)—for hyperelastic modeling. For isotropic elastomers (under uniaxial, equibiaxial, and shear deformation modes), they show that training physics‑augmented networks on equibiaxial tension yields superior cross‑mode extrapolation. In a similar spirit, we tested the extrapolation capability of three CLaNN variants trained on complex biaxial‑protocol data extracted from different regions of the specimen. We simulated two inflation scenarios: bulging of (i) a membrane homogeneous through the thickness and (ii) a thickness‑heterogeneous membrane with local thickenings acting as stress concentrators, producing nonuniform strain fields with significant shear components of the stress tensor at the pole of the specimen. The accuracy of our CLaNN solutions was assessed against a reference finite‑element solution for inflation of a Neo‑Hookean membrane with the same shear modulus used to generate the biaxial training data. With runtime comparable to the Neo‑Hookean solution, CLaNN correctly predicts displacement and stress fields during inflation of both homogeneous and thickness‑heterogeneous circular hyperelastic membranes. The absolute and relative integral errors in the norms of displacements and stresses decrease as the training data‑extraction region is expanded; toward the filleted regions of the cruciform specimen, the magnitude of the shear‑stress component increases. This agrees with the benchmark findings of \cite{tacc2024benchmarking}: as experimental coverage of the strain space increases during training of CANN/ICNN/NODE, cross‑mode extrapolation accuracy improves.

In terms of computational efficiency, CLaNN outperforms local‑interpolation methods: for a homogeneous case the speedup is about 
×
141.8
×141.8, owing to the absence of expensive k‑NN/IDW queries and external projections onto the data at each iteration. This speedup could be even higher with improved coupling between the solver and CLaNN, as well as optimized hyperparameters. For thickness heterogeneity, CLaNN remains effective without ad hoc heuristics, whereas in our previous work \cite{salamatova2020hyperelastic} the tabulated hyperelastic model required additional regularization and/or data interpolation in the vicinity of small strains. Also, because relaxation‑based equilibrium solvers often lack a clear residual‑norm stopping criterion, their runtime can increase unpredictably compared with Newton‑type methods—methods that CLaNN enables.

The study \cite{ddaniso2024} demonstrated the applicability of tabulated constitutive relations in Laplace strain measures to anisotropic biomaterials, without introducing invariants to describe anisotropy. The tabulated constitutive relation was built from synthetic experimental data generated with the Holzapfel–Gasser–Ogden model \cite{ddaniso2024} for porcine skin. Three response functions, depending on the three corresponding components of the Laplace strain tensor, were sufficient to describe the mechanical behavior of an anisotropic material in a two‑dimensional setting. This is encouraging for applying CLaNN to anisotropic materials without assumptions about material symmetries, unlike recent related ICNN‑based constitutive models built on invariants and pseudo‑invariants of the right Cauchy–Green tensor \cite{kalina2025neural}. In future work, we plan to train CLaNN on experimental data acquired from pericardial tissue tests.

In summary, relative to phenomenological hyperelastic models, CLaNN does not require assumptions about the analytical form of the potential and offers the flexibility of a universal approximator while remaining thermodynamically consistent. Relative to local interpolation approaches, CLaNN restores smoothness and convexity of the energy, recasting the equilibrium problem as a well‑conditioned minimization with predictable convergence and runtime gains.

Limitations. Our results demonstrate successful application of CLaNN to modeling a hyperelastic isotropic membrane. However, there are limitations that point to future work. First, the membrane is assumed hyperelastic. Second, we consider only isotropy in two‑dimensional settings. Going forward, we will extend CLaNN to anisotropy by approximating a tabulated constitutive relation in Laplace strain measures for an anisotropic material.

В-третьих, ...

\section{Заключение}

В итоге CLaNN объединяет механическую корректность и эффективность нейросетей: выпуклая энергия и дифференцируемость обеспечивают устойчивое 
решение вариационных задач и ускоряют расчёт по сравнению с классическими DD\textendash подходами, а также высокую способность к аппроксимации гиперупругих материалов
на малых выборках данных. 


