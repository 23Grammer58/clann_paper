

\section{Дискуссия}

В данной работе предложена и реализована физически-информированная архитектура CLaNN для гиперупругих материалов, основанная на выпуклой потенциальной энергии деформации и меры деформации Лапласа.

В исследованиях по основанной на данных механике \cite{KirchdoerferOrtiz2016,salamatova2020hyperelastic} отклик строится как локальная интерполяция (Voronoi, k-NN) в пространствах деформаций-напряжений, что ведет к невыпуклой задаче равновесия и необходимости применения релаксационных стратегий решения, вместо градиентных и второпорядковых методов со строгими гарантиями сходимости (квазиньютоновские схемы, Ньютон и т.д.). Такие методы устойчивы, но, как правило, требуют существенно большего числа шагов нагружения и внутренних итераций (а также повторяющихся k-NN/IDW-запросов), что приводит к росту времени расчёта. 

Локальную интерполяцию мы заменили интерполяцией таблично-заданного в мерах Лапласа определяющего соотношения \cite{salamatova2020hyperelastic} выпуклым по входу нейропотенциалом на базе ICNN. Такой подход априори удовлетворяет требованию к объективности, материальной симметрии \cite{freed2020laplace}, симметрии тензора напряжения и выпуклости гиперупругой модели. Последнее достигалось архитектурно, через монотонно неубывающую функцию активации и неотрицательные веса в соответствии с \cite{icnn2017}. Для построения гиперупругой модели CLaNN мы провели виртуальный аналог двухосного экспериментального исследования (по протоколу из \cite{sommer2015biomechanical}) крестообразного образца неогуковского материала. Входными данными для обучения модели CLaNN стали пары деформация-напряжение $\boldsymbol{\xi}_r^{(n)}, \mathbb{S}_r^{(n)}$, извлеченные из виртуального эксперимента. Полученная модель продемонстрировала высокую точность интерполяции для главных компонент напряжений при растяжении обучающих данных.

В работе \cite{tacc2024benchmarking} авторы сравнивают подходы к моделированию гиперупругости физически-информированными архитектурами CANN (Constitutive Artificial Neural Networks), ICNN (Input Convex Neural Networks), NODE (Neural Ordinary Differential Equations). Для изотропных эластомеров (при одноосных, равнодвухосных, сдвиговых модах дефомирования) показано, что обучение физически дополненных сетей на равнодвухосном растяжении обеспечивает лучшую межрежимную экстраполяцию. Подобным образом мы протестировали экстраполяционную способность трех вариантов CLaNN, обученных на данных сложного двухосного протокола, извлеченных из разных областей образца. Моделировалось два инфляционных сценария -- раздувание гомогенной по толщине мембраны и гетерогенной мембраны с утолщениями-концентраторами, продуцирующими неоднородные поля деформаций со значительными сдвиговыми компонентами тензора напряжений в полюсе образца.
Точность наших решений, полученная с CLaNN, сравнивалась с референтным конечно-элементным решением для раздутия неогуковской модели мембраны с тем же параметром модуля сдвига, с которым генерировались двухосные данные для обучения CLaNN. При сопоставимой с неогуковским решением скоростью CLaNN корректно предсказывает поля перемещений и напряжений при инфляции гомогенной и гетерогенной по толщине круглой гиперупругой мембраны. Абсолютные и относительные интегральные ошибки норм перемещений и напряжений падают при увеличении области извлечения данных для обучения CLaNN -- к дужкам крестообразного образца растет порядок сдвиговой компоненты напряжений. Это соответствует наблюдениям из бенчмарка \cite{tacc2024benchmarking}: с расширением покрытия экспериментальными данными пространства деформаций при обучении CANN/ICNN/NODE, растет точность межрежимной экстраполяции. 

Отметим, что по вычислительной эффективности CLaNN превосходит методы локальной интерполяции: на гомогенной задаче выигрыш составляет
порядка $\times 141.8$ за счёт отсутствия дорогостоящих k\textendash NN/IDW\textendash запросов и внешних проекций на данные на каждой итерации. 
Стоит отметить, что полученная величина выигрыша может быть и выше за счет оптимизации взаимодействия решателя и CLaNN, а также подбора оптимальных гиперпараметров.
На гетерогенной толщине CLaNN сохраняет работоспособность без специальных эвристик, 
тогда как в предыдущей работе \cite{salamatova2020hyperelastic} таблично\textendash заданная гиперупругая модель требует дополнительной регуляризации и/или интерполяции данных в окрестности малых деформаций.
Также из-за отсутствия критерия остановки расчета по норме невязки у релаксационных методов решения задачи равновесия может привести к сложнооценимому увеличению времени расчета по сравнения с ньютоновскими методами, с которыми позволяет работать CLaNN.

В работе \cite{ddaniso2024} была показана применимость таблично-заданных определяющих соотношений в мерах Лапласа для анизотропных биоматериалов. При этом инварианты для описания анизотропии не вводились. Таблично-заданное определяющее соотношение строилось по синтетическим экспериментальным данным, которые были получены с гиперупругой моделью Хользапфеля-Гассера-Огдена \cite{ddaniso2024} для свиной кожи. Трех функций отклика, зависящих от трех соответствующих компонент тензора Лапласа было достаточно для описания механического поведения анизотропного материала в двумерной постановке. Это обнадеживает применение CLaNN для анизотропного материала без введения предположений о симметриях материала, как это недавно делалось родственных работах с ICNN определяющим соотношениям, базирующимся на инвариантах и псевдо-инвариантах правого тензора деформаций Коши-Грина \cite{kalina2025neural}. В будущем мы планируем обучить модель CLaNN на экспериментальных данных, полученных при тестировании перикардиальной ткани.
Как итог, на фоне феноменологических гиперупругих моделей CLaNN не требует предположений о форме потенциала и предлагает гибкость универсального аппроксиматора потенциала при удовлетворении законов термодинамики. На фоне локально интерполяционных подходов, основанных на данных CLaNN возвращает в постановку гладкость и выпуклость энергии, что превращает задачу равновесия в хорошо обусловленную минимизацию с предсказуемой сходимостью и выигрышем по времени.
Ограничения. Наши результаты демонстрируют, что мы можем успешно применять архитектуру CLaNN для моделирования гиперупругой изотропной мембраны. Однако мы столкнулись с некоторыми ограничениями, которые указывают на будущую работу. Во-первых, в данной работе предполагается гиперупругость мембраны. Во-вторых, мы рассматриваем только изотропный случай в двумерных сценариях. В дальнейшем мы расширим применение CLaNN на анизотропию, устранив это ограничение аппроксимацией таблично-заданного определяющего соотношения в мерах Лапласа для анизотропного материала. 

В-третьих, ...

\section{Заключение}

В итоге CLaNN объединяет механическую корректность и эффективность нейросетей: выпуклая энергия и дифференцируемость обеспечивают устойчивое 
решение вариационных задач и ускоряют расчёт по сравнению с классическими DD\textendash подходами, а также высокую способность к аппроксимации гиперупругих материалов
на малых выборках данных. 


