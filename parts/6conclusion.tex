\section{Дискуссия}

В работе предложена и реализована физически информированная архитектура CLaNN для гиперупругих материалов, основанная на выпуклой
потенциальной энергии деформации и лог\textendash лапласовой параметризации кинематики. 
Для построения гиперупругой модели CLaNN мы провели виртуальные аналоги реальных двухосных экспериментальных протоколов [Biomechanical properties and microstructure of human ventricular myocardium] для крестообразных образов неогуковского материала. Входными данными для обучения модели CLaNN стали пары деформация-напряжение $\boldsymbol{\xi}_r^{(n)}, \mathbb{S}_r^{(n)}$ \cite{ddaniso2024}, извлеченные из виртуального эксперимента.
Точность решений, полученная с CLaNN, сравнивалась с референтным конечно-элементным решением для инфляции неогуковской модели мембраны с тем же параметром модуля сдвига, с которым генерировались двухосные данные для обучения CLaNN. При сопоставимом с неогуковским решением числе глобальных итераций Ньютона, модель CLaNN корректно описывает поля перемещений и напряжений при инфляции гомогенной и гетерогенной по толщине гиперупругой мембраны. 

Хотя референтное решение остается эталоном по скорости

В работах по основанной на данных механике \cite{KirchdoerferOrtiz2016,ddaniso2024} отклик строится как локальная интерполяция в пространстве деформаций-напряжений. Это не гарантирует выпуклость энергии и гладкость отклика, что ведет к невыпуклой задаче равновесия и необходимости применения релаксационных стратегий решения, вместо ньютоновских критериев остановки.

% РЫБА Родственные по кинематике решения на лог\textendash лапласовой мере применяют локальную интерполяцию функции отклика $r(\xi)$ \cite{ddaniso2024}. Наше отличие~— замена локальной интерполяции выпуклым нейропотенциалом $\psi(\xi)$ на базе ICNN с центрированием энергии и отклика, что обеспечивает $\psi(0)=0$, $S(I)=0$, выпуклость $\psi$ и положительную определённость касательных модулей (см. схему на стр.~6; раздел~1.4). На практике это улучшает сходимость Ньютона и предсказуемость расчёта

% При обучении на центральном элементе ($w{=}$1\textendash элемент) CLaNN точно интерполирует и экстраполирует растягивающие компоненты кривых нагружения: для равнодвухосного растяжения $R^2_{xx}{=}0.999$, $R^2_{yy}{=}0.999$; при валидации на неравнодвухосном протоколе~— $R^2_{xx}{=}0.993$, $R^2_{yy}{=}1.0$. При этом предсказание сдвига остаётся недостоверным ($R^2_{xy}{=}0$) из\textendash за дефицита сдвиговых состояний в центральной зоне (рис.~1.7–1.8, стр.~12–13). :contentReference[oaicite:4]{index=4}

% Критично, что расширение окна наблюдения ($w\in\{5{\times}5,\,10{\times}10,\,\text{всё поле}\}$) \emph{снижает} интегральную ошибку по полю напряжений и особенно по $S_{xy}$ в задаче инфляции (рис.~1.11–1.12). Это не «косметический» эффект: мы меняем распределение состояний деформации, включая зоны с высокими сдвигами (ближе к углам вписанного квадрата), где сдвиговые компоненты возрастают на 1–2 порядка; вместе с этим монотонно уменьшается $\|e\|_{L_2}$ и $\|e\|_{L_2,\mathrm{rel}}$ (формулы~(1.20)–(1.21), рис.~1.12). :contentReference[oaicite:5]{index=5}

% На задаче раздувания круглой мембраны (радиус $25$~мм, $p{=}5$~МПа) CLaNN воспроизводит структуру полей второго тензора Пиолы\textendash Кирхгофа $S$ для гомогенного и гетерогенного полей толщины (рис.~1.9–1.10), причём максимальные расхождения ожидаемо приходятся на $S_{xy}$ в гетерогенном случае при обучении на узком окне; по мере расширения окна ошибка падает (рис.~1.11–1.12). Это согласуется с выводами по кривым нагружения и подчёркивает, что узким местом является не архитектура, а информативность обучающей выборки по сдвигу. :contentReference[oaicite:6]{index=6}

% \begin{itemize}
%   \item \textbf{Кинематика.} Рассматривается тонкая \emph{несжимаемая} мембрана; перенос на 3D\textendash твёрдое тело с компрессией/пористостью требует отдельной адаптации и сравнения с 3D\textendash потенциалами (Ogden/Holzapfel и др.). :contentReference[oaicite:7]{index=7}
%   \item \textbf{Данные.} Все результаты получены на \emph{синтетике} (Neo\textendash Hooke); равенство по качеству референту здесь ожидаемо. Для реальных тканей преимущество универсальной аппроксимации проявится лишь при выходе за пределы классических семейств~— это нужно показать отдельно. :contentReference[oaicite:8]{index=8}
%   \item \textbf{Сдвиг.} Недостаток сдвиговых состояний в центральной зоне напрямую ограничивает точность по $S_{xy}$ во всех DD\textendash подходах (локальных и нейросетевых). Лечение одно: изменить \emph{распределение} состояний в обучении (расширить окно/протоколы), что подтверждается трендом на рис.~1.12. :contentReference[oaicite:9]{index=9}
%   \item \textbf{Скорость.} Отставание от Neo\textendash Hooke по времени решения обусловлено вычислительной ценой инференса и интерфейсом с решателем, а не принципиальными ограничениями выпуклого нейропотенциала; есть очевидный резерв оптимизации. :contentReference[oaicite:10]{index=10}
% \end{itemize}

% \subsection*{Практическая значимость для биомеханики и направления развития}

% Для инженерных задач тонких мягких оболочек (биоклапаны, сосудистые мембраны, тонкие аортальные лоскуты) CLaNN закрывает главный пробел DD\textendash подходов: \emph{термодинамически согласованная} и \emph{выпуклая} энергия при малых датасетах, что даёт предсказуемую сходимость и устойчивость при гетерогенностях (толщина/жёсткость). С практической точки зрения ближайшие шаги очевидны:
% \begin{enumerate}
%   \item \textbf{Досемплировать сдвиг.} Добавить траектории с доминирующим сдвигом (чистый сдвиг, кручение, несимметричные протоколы из табл.~1.1) и снять зависимости ``размер окна $\rightarrow$ $\|e\|_{L_2}$, $\varepsilon_{P1}$'' с доверительными интервалами. :contentReference[oaicite:11]{index=11}
%   \item \textbf{Анизотропия.} Расширить $\psi(\xi)$ на ориентированные инварианты/поля направления (например, $\psi(\xi,\mathbf{a})$) при сохранении выпуклости по $\xi$; это естественно вписывается в лог\textendash лапласову кинематику \cite{ddaniso2024}. :contentReference[oaicite:12]{index=12}
%   \item \textbf{Робастность к шуму.} Обучение на зашумлённых данных и оценка устойчивости градиента/гессиана важны для реальных экспериментов и надёжности Ньютона. :contentReference[oaicite:13]{index=13}
%   \item \textbf{3D\textendash расширение.} Проверка на тонкостенных оболочках со сжимаемостью и сопоставление с Ogden/Holzapfel в 3D\textendash тестах.
%   \item \textbf{Скорость.} JIT/ONNX\textendash экспорт, батч\textendash инференс на этапе КЭ\textendash сборки, векторизация $\partial\xi/\partial C$, использование GPU; цель~— нивелировать разрыв с параметрическими моделями. :contentReference[oaicite:14]{index=14}
%   \item \textbf{Экспериментальная валидация.} Биаксиальные испытания реальных тканей + независимые поля деформаций (например, ОКТ/ЗКИ) для внешней валидации.
% \end{enumerate}

% На фоне параметрических потенциалов CLaNN предлагает гибкость универсального аппроксиматора \emph{потенциала} при сохранении законов термодинамики. На фоне локально\textendash интерполяционных DD\textendash подходов CLaNN возвращает в постановку гладкость и выпуклость энергии, что превращает задачу равновесия в хорошо обусловленную минимизацию с предсказуемой сходимостью и выигрышем по времени (отсутствуют внешние проекции/поиски соседей). Узкое место одно и оно техническое: \emph{информативность обучающей выборки по сдвигу}; наши эксперименты показывают, как его закрыть расширением окна/протоколов (рис.~1.11–1.12; табл.~1.2). :contentReference[oaicite:15]{index=15}

\section{Заключение}

В работе предложена физически информированная архитектура CLaNN для гиперупругих материалов, основанная на выпуклой
потенциальной энергии деформации и лог\textendash лапласовой параметризации кинематики.

Архитектура обеспечивает термодинамическую согласованность; благодаря
выпуклости задача решается как гладкая выпуклая минимизация с предсказуемой сходимостью градиентных и
квазиньютоновских методов. 
При этом в архитектуре для вычисления тензора напряжений явно не используется информация о сжимаемости/несжимаемости изотропности/анизотропности
материала, что позволяет использовать CLaNN для решения задач с различными типами материалов.

В тестах интерполяции CLaNN достигает малых ошибок при наличии репрезентативных обучающих данных; в режиме
экстраполяции сохраняет устойчивость и физическую правдоподобность отклика,
тогда как локально\textendash интерполяционные DD\textendash модели (k\textendash NN/IDW) проявляют артефакты за пределами окна обучения.

На численных экспериментах раздувания закреплённой круглой мембраны (гомогенная и гетерогенная толщина) CLaNN корректно
воспроизводит поля перемещений и напряжений и демонстрирует быстрое, предсказуемое сходимостное поведение в единой
КЭ\textendash постановке для всех сравниваемых моделей.

По вычислительной эффективности CLaNN превосходит kNN\textendash основанную DD\textendash модель: на гомогенной задаче выигрыш составляет
порядка $\times 1.9$ за счёт отсутствия дорогостоящих k\textendash NN/IDW\textendash запросов и внешних проекций на данные на каждой итерации. 
Стоит отметить, что полученная величина выигрыша может быть и выше за счет оптимизации взаимодействия решателя и CLaNN, а также подбора оптимальных гиперпараметров.
На гетерогенной толщине CLaNN сохраняет работоспособность без специальных эвристик, 
тогда как DD\textendash модель требует дополнительной регуляризации и/или интерполяции данных в окрестности малых деформаций.
Также из-за отсутсвия критерия остановки расчета по норме невязки у релаксационных методов решения задачи равновесия может привести к сложноценимому увеличению времени расчета
по сравнения с ньютоновскими методами, с которыми позволяет работать CLaNN.

В итоге CLaNN объединяет механическую корректность и эффективность нейросетей: выпуклая энергия и дифференцируемость обеспечивают устойчивое 
решение вариационных задач и ускоряют расчёт по сравнению с классическими DD\textendash подходами, а также высокую способность к аппроксимации гиперупругих материалов
на малых выборках данных. 
Дальнейшие направления: протестировать анизотропные материалы и реальные экспериментальные данные.


