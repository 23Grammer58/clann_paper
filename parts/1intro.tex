\section{Введение}

Математические модели, способные предсказывать нелинейное механическое поведение мягких материалов при больших деформациях, требуются в широком спектре инженерных отраслей -- от полимерной промышленности, до робототехники и персонализированной медицины \cite{
Mechanical characterization and FE modelling of a hyperelastic material
Quantifying the uncertainty in a hyperelastic soft tissue model with stochastic parameters
Control-oriented models for hyperelastic soft robots through differential geometry of curves
}
Основой таких моделей служит нелинейная теория упругости \cite{A Comparative Study of Several Material Models for Prediction of Hyperelastic Properties: Application to Silicone-Rubber and Soft Tissues}, где зависимость тензора напряжений от переменных, характеризующих кинематику материала, описывается так называемыми определяющими соотношениями, или уравнениями состояния \cite{Nonlinear solid mechanics: a continuum approach for engineering science}. 
При моделировании напряженно-деформированного состояния полимеров и биологических тканей широко распространены гиперупругие определяющие соотношения \cite{
Hyperelastic structures: A review on the mechanics and biomechanics}.

В гиперупругой постановке постулируется существование упругого потенциала $\psi$, зависящего от выбранной меры деформации, который полностью описывает механическое поведение материала. 
При этом он должен удовлетворять ряду требований: отражать материальную симметрию, не зависеть от выбранной системы отсчета, обладать свойствами поливыпуклости \cite{Convexity conditions and existence theorems in nonlinear elasticity}, что является достаточным условием существования решений краевых задач гиперупругости \cite{Mathematical elasticity: Three-dimensional elasticity
Hyperelastic membrane modelling based on data-driven constitutive relations}.

Для мягких материалов предложено множество гиперупругих моделей \cite{Hyperelastic energy densities for soft biological tissues: a review}, большинство из которых удовлетворяют требованиям к материальной симметрии, объективности и поливыпуклости благодаря инвариантному подходу. 
Это означает, что для выбранной меры деформации задается набор инвариантов, а упругий потенциал является функцией этих инвариантов. 
Обычной практикой является использование инвариантов правых/левых тензоров деформации Коши-Грина. Для изотропных материалов упругий потенциал может быть выражен как функция от трех инвариантов правого тензора деформации Коши-Грина $\psi = \psi_{vol}(J) + \psi_{iso}(I_1,I_2,I_3)$, где $J$ -- якобиан, выражающий изменение объема тела при деформации, $I_1, I_2, I_3$ -- инварианты правого тензора деформации Коши-Грина. 
Дальнейшим расширением инвариантного подхода является введение так называемых псевдоинвариантов правого тензора Коши-Грина $I_4,...I_8$, позволяющих описывать классы трансверсально-изотропных и ортотропных материалов. \cite{Nonlinear solid mechanics: a continuum approach for engineering science}.
%Для несжимаемого трансверсально-изотропного материала упругий потенциал задается как $\psi = \psi_{iso}(I_1,I_2,I_3) + \psi_{aniso}(I_4,I_5)$ .
%, где $I_4 = \mathbf{a_0} (\mathbf{F}^{\mathrm{T}} \mathbf{F})\mathbf{a_0}$, $I_5 = \mathbf{a_0} (\mathbf{F}^{\mathrm{T}} \mathbf{F})^2 \mathbf{a_0}$ 
Такой подход требует априорного задания упругого потенциала аналитической функцией с параметрами, которые определяются из экспериментальных данных. Основными недостатками этого подхода являются неединственность оптимального набора параметров модели, отсутствие у инвариантов прямого физического смысла в терминах деформации \cite{On the use of the upper triangular (or QR) decomposition for developing constitutive equations for Green-elastic materials} с вытекающим из этого требованием к натурному эксперименту, а именно, достижение однородности деформаций и напряжений при механическом исследовании тестировании материала, субъективность выбора формы потенциала из множества построенных экспертами моделей \cite{Interpretable data-driven modeling of hyperelastic membranes}.

В некоторой степени, эти недостатки устраняют конструированием наилучшей гиперупругой модели регрессионными методами из набора априорно заданных мономов на основе инвариантов \cite{A new family of Constitutive Artificial Neural Networks towards automated model discovery}, или редукцией обобщенных моделей на основе информационного анализа экспериментальных данных \cite{On the AIC-based model reduction for the general Holzapfel–Ogden myocardial constitutive law}. 
В совокупности с полнополевыми методами оценки экспериментальных деформаций (цифровая корреляция избражений DIC \cite{High-speed 3D digital image correlation vibration measurement: Recent advancements and noted limitations}), методами виртуальных полей VFM \cite{VFM} и inverse FE \cite{NN-Euclid}, это становится мощным инструментом моделирования механики материалов в рамках гиперупругости. 
Однако, такие подходы остаются феноменологическими и все так же требуют экспертный выбор модели.

% Важное преимущество гиперупругой постановки состоит в том, что она не требует знания аналитического вида упругого потенциала. 
Для построения гиперупругой модели необязательно знать аналитический вид (форму) упругого потенциала, что является преимуществом, обеспечивающим возможность построения модели на основе данных.
Для задания определяющих соотношений в случае гиперупругого материала, достаточно знать производные упругого потенциала по выбранной мере деформации, так называемые функции отклика \cite{Nonlinear solid mechanics: a continuum approach for engineering science}.
С применением полнополевых методов оценки экспериментальных деформаций DIC и напряжений \cite{Numerical_study_of_stress_estimation_methods_for_membrane_inflation
In vitro analysis of localized aneurysm rupture}, функции отклика могут быть построены напрямую на основе экспериментальных данных, полученных при тестировании материала в широком диапазоне различных режимов деформации. Это стимулирует исследование подходов к построению гиперупругих моделей, основанных на данных \cite{Data-driven computational mechanics}. 

В работах \cite{Interpretable data-driven modeling of hyperelastic membranes
Data-Driven Anisotropic Biomembrane Simulation Based on the Laplace Stretch} предлагается метод прямого моделирования механики изотропных и анизотропных материалов, основанного на данных, с использованием функций отклика, основанных на физически интерпретируемой мере деформации Лапласа\cite{On the use of the upper triangular (or QR) decomposition for developing constitutive equations for Green-elastic materials
Laplace stretch: Eulerian and Lagrangian formulations}, в котором обходят проблемы инвариантной формулировки гиперупругой модели, напрямую строя функции отклика на основе экспериментальных данных. 
При этом не требуются какие-либо предварительные знания о симметрии материала. 
Совокупность функций отклика формирует таблично-заданное определяющее соотношение.
Нелинейная система алгебраических уравнений для виртуального квазистатического растяжения и раздутия материалов в этих случаях решается простым методом релаксации, где метод интерполяции обратного взвешенного расстояния находит требуемые значения функций отклика на каждой итерации в любой точке пространства деформаций Лапласа. 
Ограничениями такого подхода являются требования к "богатству" данных и невозможность применения градиентных методов решения нелинейных систем алгебраических уравнений в силу негладкости интерполяции таблично-заданного определяющего соотношения и отсутствия гарантий выпуклости энергии.

Параллельно с этим развиваются физически-информированные нейросетевые подходы. 
В частности, при использовании выпуклой по входу нейронной сети (ICNN) \cite{Input Convex Neural Networks} и монотонного неубывания энергиии деформации относительно инвариантов можно гарантировать поливыпклость \cite{boyd2004convex}, тем самым удовлетворяя требованиям к гиперупругим потенциалам \cite{Benchmarking physics-informed frameworks for data-driven hyperelasticity}. 
В работе \cite{NEURAL NETWORKS MEET ANISOTROPIC HYPERELASTICITY: A FRAMEWORK BASED ON GENERALIZED STRUCTURE TENSORS AND ISOTROPIC TENSOR FUNCTIONS} показана инвариантная архитектура физически-информированной нейронной сети, совместимая с конечно-элементными пакетами. 
Несмотря на интерпретируемость и термодинамическую корректность, архитектура сети включает в себя набор предположений -- обобщенных структурных тензоров \cite{Ebbing Phd}, что фактически фиксирует класс симметрии материала.

В рамках данной работы мы предлагаем подход, который объединяет преимущества представления гиперупругой модели таблично-заданным определяющим соотношением в мерах деформаций Лапласа \cite{Data-Driven Anisotropic Biomembrane Simulation Based on the Laplace Stretch} и физически дополненных нейронных сетей на основе ICNN \cite{Input Convex Neural Networks}, удовлетворяющих требованиям к гиперупругим моделям механики материалов. Мы формулируем термодинамически корректный, объективный по построению, выпуклый по входу, и не требующий знаний о симметрии материала гиперупругий потенциал, %%в котором анизотропия восстанавливается непосредственно из данных.
Гладкость аппроксимации обеспечивает совместимость с градиентными методами решения систем нелинейных алгебраических уравнений. В сравнении с таблично-заданными определяющими соотношениями, CLaNN снимает ограничения, связанные с дискретностью аппроксимации, сохраняя интерпретируемость мер деформации и повышая устойчивость экстраполяции.


\paragraph{Организация статьи}


