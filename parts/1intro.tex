\section{Введение}

Математические модели, способные предсказывать нелинейное механическое поведение мягких материалов при больших деформациях, требуются в широком спектре инженерных отраслей -- от полимерной промышленности, до робототехники и персонализированной медицины \cite{shahzad2015mechanical,hauseux2018quantifying,caasenbrood2023control}.
Основой таких моделей служит нелинейная теория упругости \cite{martins2006comparative}, где зависимость тензора напряжений от переменных, характеризующих кинематику материала, описывается так называемыми определяющими соотношениями, или уравнениями состояния \cite{holzapfel2000nonlinear}. 
При моделировании напряженно-деформированного состояния полимеров и биологических тканей широко распространены гиперупругие определяющие соотношения \cite{khaniki2023hyperelastic}.
В гиперупругой постановке постулируется существование упругого потенциала $\psi$, зависящего от выбранной меры деформации, который полностью описывает механическое поведение материала. 
При этом он должен удовлетворять ряду требований: отражать материальную симметрию, не зависеть от выбранной системы отсчета, обладать свойствами поливыпуклости \cite{ball1976convexity}, что является достаточным условием существования решений краевых задач гиперупругости \cite{ciarlet2021mathematical}.

Для мягких материалов предложено множество гиперупругих моделей \cite{chagnon2015hyperelastic}, большинство из которых удовлетворяют требованиям к материальной симметрии, объективности и поливыпуклости благодаря инвариантному подходу. 
Это означает, что для выбранной меры деформации задается набор инвариантов, а упругий потенциал является функцией этих инвариантов. 
Обычной практикой является использование инвариантов правых/левых тензоров деформации Коши-Грина. Для изотропных материалов упругий потенциал может быть выражен как функция от трех инвариантов правого тензора деформации Коши-Грина $\psi = \psi_{vol}(J) + \psi_{iso}(I_1,I_2,I_3)$, где $J$ -- якобиан, выражающий изменение объема тела при деформации, $I_1, I_2, I_3$ -- инварианты правого тензора деформации Коши-Грина. 
Дальнейшим расширением инвариантного подхода является введение так называемых псевдоинвариантов правого тензора Коши-Грина $I_4,...I_8$, позволяющих описывать классы трансверсально-изотропных и ортотропных материалов. \cite{holzapfel2000nonlinear}.

Такой подход требует априорного задания упругого потенциала аналитической функцией с параметрами, которые определяются из экспериментальных данных. Основными недостатками этого подхода являются неединственность оптимального набора параметров модели, отсутствие у инвариантов прямого физического смысла в терминах деформации \cite{srinivasa2012use} с вытекающим из этого требованием к натурному эксперименту, а именно, достижение однородности деформаций и напряжений при механическом исследовании тестировании материала, субъективность выбора формы потенциала из множества построенных экспертами моделей \cite{salamatova2023interpretable}.

В некоторой степени, эти недостатки устраняют конструированием наилучшей гиперупругой модели регрессионными методами из набора априорно заданных мономов на основе инвариантов \cite{linka2023new}, или редукцией обобщенных моделей на основе информационного анализа экспериментальных данных \cite{guan2019aic}. 
В совокупности с полнополевыми методами оценки экспериментальных деформаций (цифровая корреляция избражений DIC \cite{beberniss2017high}), методами виртуальных полей VFM \cite{pierron2012virtual} и inverse FE \cite{nguyen2011inverse}, это становится мощным инструментом моделирования механики материалов в рамках гиперупругости. 
Однако, такие подходы остаются феноменологическими и все так же требуют экспертный выбор модели.

Для построения гиперупругой модели необязательно знать аналитический вид (форму) упругого потенциала, что является преимуществом, обеспечивающим возможность построения модели на основе данных.
Для задания определяющих соотношений в случае гиперупругого материала, достаточно знать производные упругого потенциала по выбранной мере деформации, так называемые функции отклика \cite{holzapfel2000nonlinear}.
С применением полнополевых методов оценки экспериментальных деформаций DIC и напряжений \cite{romo2014vitro}, функции отклика могут быть построены напрямую на основе экспериментальных данных, полученных при тестировании материала в широком диапазоне различных режимов деформации. Это стимулирует исследование подходов к построению гиперупругих моделей, основанных на данных \cite{KirchdoerferOrtiz2016}. 

В работах \cite{xi2023} предлагается метод прямого моделирования механики изотропных и анизотропных материалов, основанного на данных, с использованием функций отклика, основанных на физически интерпретируемой мере деформации Лапласа\cite{freed2020laplace}, в котором обходят проблемы инвариантной формулировки гиперупругой модели, напрямую строя функции отклика на основе экспериментальных данных. 
При этом не требуются какие-либо предварительные знания о симметрии материала. 
Совокупность функций отклика формирует таблично-заданное определяющее соотношение.
Нелинейная система алгебраических уравнений для виртуального квазистатического растяжения и раздутия материалов в этих случаях решается простым методом релаксации, где метод интерполяции обратного взвешенного расстояния находит требуемые значения функций отклика на каждой итерации в любой точке пространства деформаций Лапласа. 
Ограничениями такого подхода являются требования к "богатству" данных и невозможность применения градиентных методов решения нелинейных систем алгебраических уравнений в силу негладкости интерполяции таблично-заданного определяющего соотношения и отсутствия гарантий выпуклости энергии.

Параллельно с этим развиваются физически-информированные нейросетевые подходы. 
В частности, при использовании выпуклой по входу нейронной сети (ICNN) \cite{icnn2017} и монотонного неубывания энергиии деформации относительно инвариантов можно гарантировать поливыпуклость \cite{boyd2004convex}, тем самым удовлетворяя требованиям к гиперупругим потенциалам \cite{tacc2024benchmarking}. 
В работе \cite{kalina2025neural} показана инвариантная архитектура физически-информированной нейронной сети, совместимая с конечно-элементными пакетами. 
Несмотря на интерпретируемость и термодинамическую корректность, архитектура сети включает в себя набор предположений -- обобщенных структурных тензоров \cite{schroder2008anisotropic}, что фактически фиксирует класс симметрии материала.

В рамках данной работы мы предлагаем подход, который объединяет преимущества представления гиперупругой модели таблично-заданным определяющим соотношением в мерах деформаций Лапласа \cite{ddaniso2024} и физически дополненных нейронных сетей на основе ICNN \cite{icnn2017}, удовлетворяющих требованиям к гиперупругим моделям механики материалов. Мы формулируем термодинамически корректный, объективный по построению, выпуклый по входу, и не требующий знаний о симметрии материала гиперупругий потенциал.
Гладкость аппроксимации обеспечивает совместимость с градиентными методами решения систем нелинейных алгебраических уравнений. В сравнении с таблично-заданными определяющими соотношениями, CLaNN снимает ограничения, связанные с дискретностью аппроксимации, сохраняя интерпретируемость мер деформации и повышая устойчивость экстраполяции.

\paragraph{Область применения и ограничения}
В настоящей работе мы ограничиваемся квазистатическими задачами для тонких мембран, описываемых двумерной мембранной формулировкой; динамические эффекты и общая трёхмерная постановка выходят за рамки исследования.

\paragraph{Организация статьи}
В разделе «Кинематика» вводятся обозначения и мера деформации Лапласа, раздел «Напряжение и термодинамическая корректность» связывает меру деформации с расчётом напряжений, раздел «Архитектура» описывает CLaNN и её производные, раздел «Результаты» собирает виртуальные эксперименты, а разделы «Дискуссия» и «Заключение» обобщают выводы и намечают дальнейшие исследования.
