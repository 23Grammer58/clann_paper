\section{Conclusion}

In this work we propose and implement a physics‑augmented CLaNN architecture for hyperelastic materials, based on a convex strain‑energy potential and the Laplace strain measure.

In data‑driven computational mechanics \cite{KirchdoerferOrtiz2016,salamatova2020hyperelastic}, the material response is often constructed via local interpolation (Voronoi, k‑NN) in strain–stress spaces. This leads to a nonconvex equilibrium problem and necessitates relaxation‑type solution strategies instead of gradient‑based and second‑order methods with strong convergence guarantees (quasi‑Newton schemes, Newton–Raphson, etc.). Such relaxation methods are robust but typically require many more load increments and internal iterations (as well as repeated k‑NN/IDW queries), which increases runtime.

We replace local interpolation of a tabulated constitutive relation in Laplace strain measures \cite{salamatova2020hyperelastic} with an input‑convex neural potential based on ICNN. This approach a priori satisfies objectivity, material symmetry \cite{freed2020laplace}, stress‑tensor symmetry, and convexity of the hyperelastic model. The latter is enforced architecturally via a monotonically nondecreasing activation function and nonnegative weights, in accordance with \cite{icnn2017}. To build the CLaNN hyperelastic model, we performed a virtual analog of a biaxial experimental study (following \cite{sommer2015biomechanical}) on a cruciform specimen of a Neo‑Hookean material. The training inputs for CLaNN were strain–stress pairs, $\boldsymbol{\xi}_r^{(n)}, \mathbb{S}_r^{(n)}$, extracted from the virtual experiment. The resulting model exhibited high interpolation accuracy for the principal stress components under the tensile states included in the training set.

In \cite{tacc2024benchmarking}, the authors compare physics‑augmented architectures—CANN (Constitutive Artificial Neural Networks), ICNN (Input Convex Neural Networks), and NODE (Neural Ordinary Differential Equations)—for hyperelastic modeling. For isotropic elastomers (under uniaxial, equibiaxial, and shear deformation modes), they show that training physics‑augmented networks on equibiaxial tension yields superior cross‑mode extrapolation. In a similar spirit, we tested the extrapolation capability of three CLaNN variants trained on complex biaxial‑protocol data extracted from different regions of the specimen. We simulated two inflation scenarios: bulging of (i) a membrane homogeneous through the thickness and (ii) a thickness‑heterogeneous membrane with local thickenings acting as stress concentrators, producing nonuniform strain fields with significant shear components of the stress tensor at the pole of the specimen. The accuracy of our CLaNN solutions was assessed against a reference finite‑element solution for inflation of a Neo‑Hookean membrane with the same shear modulus used to generate the biaxial training data. With runtime comparable to the Neo‑Hookean solution, CLaNN correctly predicts displacement and stress fields during inflation of both homogeneous and thickness‑heterogeneous circular hyperelastic membranes. The absolute and relative integral errors in the norms of displacements and stresses decrease as the training data‑extraction region is expanded; toward the filleted regions of the cruciform specimen, the magnitude of the shear‑stress component increases. This agrees with the benchmark findings of \cite{tacc2024benchmarking}: as experimental coverage of the strain space increases during training of CANN/ICNN/NODE, cross‑mode extrapolation accuracy improves.

In terms of computational efficiency, CLaNN outperforms local‑interpolation methods: for a homogeneous case the speedup is about ×141.8, owing to the absence of expensive k‑NN/IDW queries and external projections onto the data at each iteration. This speedup could be even higher with improved coupling between the solver and CLaNN, as well as optimized hyperparameters. For thickness heterogeneity, CLaNN remains effective without ad hoc heuristics, whereas in our previous work \cite{salamatova2020hyperelastic} the tabulated hyperelastic model required additional regularization and/or data interpolation in the vicinity of small strains. Also, because relaxation‑based equilibrium solvers often lack a clear residual‑norm stopping criterion, their runtime can increase unpredictably compared with Newton‑type methods—methods that CLaNN enables.

The study \cite{ddaniso2024} demonstrated the applicability of tabulated constitutive relations in Laplace strain measures to anisotropic biomaterials, without introducing invariants to describe anisotropy. The tabulated constitutive relation was built from synthetic experimental data generated with the Holzapfel–Gasser–Ogden model \cite{ddaniso2024} for porcine skin. Three response functions, depending on the three corresponding components of the Laplace strain tensor, were sufficient to describe the mechanical behavior of an anisotropic material in a two‑dimensional setting. This is encouraging for applying CLaNN to anisotropic materials without assumptions about material symmetries, unlike recent related ICNN‑based constitutive models built on invariants and pseudo‑invariants of the right Cauchy–Green tensor \cite{kalina2025neural}. In future work, we plan to train CLaNN on experimental data acquired from pericardial tissue tests.

In summary, relative to phenomenological hyperelastic models, CLaNN does not require assumptions about the analytical form of the potential and offers the flexibility of a universal approximator while remaining thermodynamically consistent. Relative to local interpolation approaches, CLaNN restores smoothness and convexity of the energy, recasting the equilibrium problem as a well‑conditioned minimization with predictable convergence and runtime gains.

Limitations. Our results demonstrate successful application of CLaNN to modeling a hyperelastic isotropic membrane. However, there are limitations that point to future work. First, the membrane is assumed hyperelastic. Second, we consider only isotropy in two‑dimensional settings. Going forward, we will extend CLaNN to anisotropy by approximating a tabulated constitutive relation in Laplace strain measures for an anisotropic material.

We proposed a physics-augmented CLaNN architecture for hyperelastic materials, based on a convex
strain energy potential and log\textendash Laplace kinematic parameterization.

The architecture ensures thermodynamic consistency; thanks to
convexity, the problem is solved as a smooth convex minimization with predictable convergence of gradient and
quasi-Newton methods.
At the same time, the architecture for stress computation does not explicitly use information about compressibility/incompressibility or isotropy/anisotropy
of the material, which allows CLaNN to be applied to problems with different material types.

In interpolation tests CLaNN achieves small errors given representative training data; in the extrapolation regime it maintains stability and physically plausible response,
whereas locally interpolatory DD models (k\textendash NN/IDW) exhibit artifacts outside the training window.

In numerical experiments of inflating a clamped circular membrane (homogeneous and heterogeneous thickness), CLaNN accurately
reproduces displacement and stress fields and exhibits fast, predictable convergence within a unified
FE formulation for all models compared.

In terms of computational efficiency, CLaNN outperforms the kNN\textendash based DD model: in the homogeneous case the speedup is
about $\times 1.9$ due to the absence of costly k\textendash NN/IDW queries and external projections onto data at each iteration.
We note that the obtained speedup can be even higher with improved coupling between the solver and CLaNN, as well as optimal hyperparameter tuning.
For heterogeneous thickness CLaNN remains operational without special heuristics,
whereas the DD model requires additional regularization and/or data interpolation near small strains.
Also, the lack of a residual-based stopping criterion in relaxation methods for equilibrium problems can lead to a difficult-to-estimate increase in runtime
relative to Newton-type methods supported by CLaNN.

In summary, CLaNN combines mechanical soundness with the efficiency of neural networks: convex energy and differentiability provide stable 
solutions to variational problems and accelerate computation compared with classical DD approaches, and show high approximation capability for hyperelastic materials
on small datasets.
Future work includes testing on anisotropic materials and real experimental data.

Author Contributions
Conceptualization, D.D. and V.S.; methodology, D.D.; software, D.D. and A.L.; numerical experiments, A.O; investigation, D.D. and A.O.; draft preparation D.D. and A.O.; supervision, V.S. All authors have read and agreed to the published version of the manuscript.
Funding: The work was supported by the Russian Science Foundation through the grant No. 24-21-20075
Institutional Review Board Statement
Not applicable.
Informed Consent Statement
Not applicable.
Data Availability Statement
Conflicts of Interest
The authors declare no conflict of interest.

