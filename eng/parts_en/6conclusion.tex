\section{Conclusion}

We proposed a physics-informed CLaNN architecture for hyperelastic materials, based on a convex
strain energy potential and log\textendash Laplace kinematic parametrization.

The architecture ensures thermodynamic consistency; thanks to
convexity, the problem is solved as a smooth convex minimization with predictable convergence of gradient and
quasi-Newton methods.
At the same time, the architecture for stress computation does not explicitly use information about compressibility/incompressibility or isotropy/anisotropy
of the material, which allows CLaNN to be applied to problems with different material types.

In interpolation tests CLaNN achieves small errors given representative training data; in the extrapolation regime it maintains stability and physically plausible response,
whereas locally interpolatory DD models (k\textendash NN/IDW) exhibit artifacts outside the training window.

In numerical experiments of inflating a clamped circular membrane (homogeneous and heterogeneous thickness), CLaNN accurately
reproduces displacement and stress fields and exhibits fast, predictable convergence within a unified
FE formulation for all models compared.

In terms of computational efficiency, CLaNN outperforms the kNN\textendash based DD model: on the homogeneous case the gain is
about $\times 1.9$ due to the absence of costly k\textendash NN/IDW queries and external projections onto data at each iteration.
We note that the obtained speedup can be even higher with improved coupling between the solver and CLaNN, as well as optimal hyperparameter tuning.
For heterogeneous thickness CLaNN remains operational without special heuristics,
whereas the DD model requires additional regularization and/or data interpolation near small strains.
Also, the lack of a residual-based stopping criterion in relaxation methods for equilibrium problems can lead to a difficult-to-estimate increase in runtime
relative to Newton-type methods supported by CLaNN.

In summary, CLaNN combines mechanical soundness with the efficiency of neural networks: convex energy and differentiability provide stable 
solutions to variational problems and accelerate computation compared with classical DD approaches, and show high approximation capability for hyperelastic materials
on small datasets.
Future work: test anisotropic materials and real experimental data.


