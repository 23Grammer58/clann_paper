\section{Kinematics}
\textbf{Basic relations}

We consider the equilibrium of a thin incompressible hyperelastic membrane of thickness $H$ under
prescribed loads.
The membrane deformation is characterized by the deformation of its midsurface.
Let \(\vect{X}\) and \(\vect{x}\) denote point positions,
in the corresponding bases \(\vect{E}_{\alpha}\) and \(\vect{e}_{\alpha}\),
in the reference (undeformed) \(\Omega_0 \subset \mathbb{R}^2\) and current (deformed) \(\Omega_t \subset \mathbb{R}^2\)
configurations of the membrane surface, respectively.
The deformation is defined by the mapping \(\vect{x} = \vect{x}(\vect{X})\),
the surface deformation gradient is \(\mathbb{F} = \vect{e}_{\alpha}\otimes\vect{E}^{\alpha}\),
and the right Cauchy–Green tensor is \(\mathbb C = C_{\alpha\beta} \,\vect{e}_{\alpha} \otimes \vect{e}_{\beta} = \mathbb F^{\top} \mathbb F\).
To define the strain measure we use the Laplace measure \(\vect{\xi} = (\xi_1 , \xi_2 , \xi_3)^T\) \cite{xi2023},
which may be computed in two equivalent ways:
either via the QR decomposition of the deformation gradient \(\mathbb F = \vect Q \vect R\) with \(\tilde{\mathbb F} = \vect R\),
or via the Cholesky factorization of the right Cauchy–Green tensor \(\mathbb C = \tilde{\mathbb F}^{\top}\tilde{\mathbb F}\) (Appendix \ref{app:cholesky}).
In this case the hyperelastic strain energy is a function of the Laplace strain, \(\psi = \psi(\vect{\xi})\).


\textbf{Laplace strain measure}
In two dimensions we introduce
\begin{equation}
\xi_1 = \ln(\tilde f_{11}),\quad \xi_2 = \ln(\tilde f_{22}),\quad \xi_3 = \frac{\tilde f_{12}}{\tilde f_{11}}, 
\quad \tilde{\mathbb F} = \tilde f_{\alpha\beta} \; \vect{e}_{\alpha} \otimes \vect{e}_{\beta}.
\label{eq:laplace_coords}
\end{equation}


