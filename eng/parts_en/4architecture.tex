\section{CLaNN architecture and its derivatives}

Within the proposed CLaNN (Convex Laplace Neural Network) framework,
the strain energy \(\psi(\vect{\xi})\) with the Laplace strain measure is approximated
by an input convex neural network (ICNN) \cite{icnn2017}, and the second Piola--Kirchhoff
stress \(\mathbb{S}\) is computed using the explicit expression \eqref{eq:stress_components_2d}. 

\textbf{General ICNN architecture}

ICNN is a class of feed-forward neural networks whose output is (jointly) convex with respect to a distinguished
subset of inputs \cite{icnn2017}. Following \cite{icnn2017}, a function
$f:\mathbb{R}^{n_x}\times\mathbb{R}^{n_y}\rightarrow\mathbb{R}$ represented by a neural network is called
\emph{input-convex} in $x$ if, for all $x_1,x_2\in\mathbb{R}^{n_x}$, $y\in\mathbb{R}^{n_y}$ and $\lambda\in[0,1]$,
it satisfies
\begin{equation}
  f(\lambda x_1 + (1-\lambda)x_2,\,y)
  \;\le\;
  \lambda f(x_1,y) + (1-\lambda) f(x_2,y).
\end{equation}
In our setting we do not distinguish parameters $y$ and consider only the strain variables
$\boldsymbol{\xi}\in\mathbb{R}^3$, so that the strain energy $\psi:\mathbb{R}^3\to\mathbb{R}$ is convex if for all
$\vect{\xi}_1, \vect{\xi}_2 \in \mathbb{R}^3$ and $\lambda \in [0,1]$ Jensen's inequality holds
\cite{BoydVandenberghe2004}:
\begin{equation}
  \psi(\lambda \vect{\xi}_1 + (1-\lambda) \vect{\xi}_2)
  \;\le\;
  \lambda \psi(\vect{\xi}_1) + (1-\lambda) \psi(\vect{\xi}_2).
\label{eq:convexity_definition}
\end{equation}

The standard ICNN architecture \cite{icnn2017} is constructed as follows.
Let $\boldsymbol{\xi}\in\mathbb{R}^3$ be the input, $z^{(0)}=\boldsymbol{0}$ be the initial hidden state, and for
layers $\ell=0,\dots,L-1$ define
\begin{equation}
  z^{(\ell+1)} =
  \varphi\!\big(\mathbf{W}_z^{(\ell)} z^{(\ell)} + \mathbf{W}_x^{(\ell)} \boldsymbol{\xi} + \mathbf{b}^{(\ell)}\big),
\end{equation}
where $z^{(\ell)}\in\mathbb{R}^{h_\ell}$, $\mathbf{W}_z^{(\ell)}\in\mathbb{R}^{h_{\ell+1}\times h_\ell}$,
$\mathbf{W}_x^{(\ell)}\in\mathbb{R}^{h_{\ell+1}\times 3}$, and $\mathbf{b}^{(\ell)}\in\mathbb{R}^{h_{\ell+1}}$.
The scalar output is then given by an affine readout
\begin{equation}
  \psi(\boldsymbol{\xi}) = a^{\top} z^{(L)} + c,
\end{equation}
with $a\in\mathbb{R}^{h_L}$ and $c\in\mathbb{R}$. The key structural conditions guaranteeing convexity with respect
the input $\boldsymbol{\xi}$ are \cite{icnn2017}:
(i) elementwise convex, monotonically nondecreasing activation $\varphi$;
(ii) nonnegative weights on the $z\!\to\!z$ connections, $\mathbf{W}_z^{(\ell)}\!\ge\!0$ for all $\ell$
(no sign constraints on $\mathbf{W}_x^{(\ell)}$ and $\mathbf{b}^{(\ell)}$);
(iii) a direct affine connection from the input $\boldsymbol{\xi}$ to every hidden layer as in the formula above;
(iv) nonnegative output weights $a\!\ge\!0$ (or, equivalently, an additional nondecreasing activation layer at the
output). Under these assumptions the network output $\psi(\boldsymbol{\xi})$ is a convex function of the strain
variables $\boldsymbol{\xi}$.

\textbf{Step 1. One-layer ICNN and activation choice.}
Consider a one-hidden-layer ICNN for approximating $\psi(\boldsymbol{\xi})$:
\begin{equation}
  s = \mathbf{W}_1 \,\boldsymbol{\xi} + \mathbf{b}_1,\qquad
  z = \varphi_{\beta}^2(s),\qquad
  \tilde{\psi} = \mathbf{W}_2^{\top} z + b_2,\qquad \mathbf{W}_2 \ge 0.
  \label{eq:icnn_onelayer}
\end{equation}

\begin{equation}
  \varphi_{\beta}(x) = \frac{\operatorname{softplus}(\beta x)}{\beta},
  \label{eq:softplus_activation}
\end{equation}

Here $\varphi_{\beta}$ is a convex nondecreasing activation \cite{dugas2001incorporating},
which smoothly approximates ReLU and for finite $\beta$ is strictly convex; $\varphi_{\infty}(x)=\max(0,x)$.
The constraint $\mathbf{W}_2\!\ge 0$ preserves convexity of the linear combination. 
Dimensions: $\mathbf{W}_1\!\in\mathbb{R}^{h\times 3}$, $\mathbf{b}_1\!\in\mathbb{R}^{h}$, 
$\mathbf{W}_2\!\in\mathbb{R}^{h}_{\ge 0}$, $h$ is the hidden layer width.

\textbf{Step 2. Centering the energy $\psi$ at the natural state.}
To satisfy $\psi(\mathbf{0})=0$, we center the energy by
subtracting the value of the nonlinear part at $\boldsymbol{\xi}=\mathbf{0}$:
\begin{equation}
  z_0 = \varphi_{\beta}^2(\mathbf{b}_1),\qquad
  \psi(\boldsymbol{\xi}) = \mathbf{W}_2^{\top}\big(z - z_0\big),\qquad (b_2 \equiv 0).
  \label{eq:center_psi}
\end{equation}
Then $\psi(\mathbf{0})=0$. Since $z_0$ does not depend on $\boldsymbol{\xi}$, 
the gradient $\partial\psi/\partial\boldsymbol{\xi}$ and the Hessian $\partial^2\psi/\partial\boldsymbol{\xi}^2$ 
coincide with those of $\tilde{\psi}$, preserving convexity and smoothness.

\textbf{Step 3. Centering the response $\mathbf{r}$ at the natural configuration.}
To satisfy $\mathbb S(\mathbb I)=\vect 0$, we set the linear response to zero at 
$\boldsymbol{\xi}=\mathbf{0}$:
\begin{equation}
  \mathbf{r}_0 := \frac{\partial \psi}{\partial \boldsymbol{\xi}}\bigg|_{\boldsymbol{\xi}=\mathbf{0}},\qquad
  \psi_{\mathrm{phys}}(\boldsymbol{\xi}) = \psi(\boldsymbol{\xi}) - \mathbf{r}_0^{\top}\boldsymbol{\xi}.
  \label{eq:phys_energy}
\end{equation}

Then $\psi_{\mathrm{phys}}(\mathbf{0})=0$ and $\mathbf{r}(\mathbf{0})=\mathbf{0}$, 
and by the chain rule \eqref{eq:chain-rule} we obtain $\mathbf{S}(\mathbf{I})=\mathbf{0}$. 
Subtracting the linear term does not change the Hessian and preserves convexity. 
Since $\mathbf{r}(\mathbf{0})=\mathbf{0}$, the point $\boldsymbol{\xi}=\mathbf{0}$ is a minimizer of 
$\psi_{\mathrm{phys}}$, hence $\psi_{\mathrm{phys}}\ge 0$.

After constructing $\psi_{\mathrm{phys}}$,  
$\partial\psi/\partial\boldsymbol{\xi}$ is computed by automatic differentiation (autodiff) implemented in modern machine learning libraries (e.g., PyTorch, TensorFlow, JAX) 
\cite{pytorch2019,tensorflow2016,jax2018}, 
after which the stress tensor $\mathbb{S}$ is obtained via \eqref{eq:stress_components_2d} using the relation $\psi(\mathbb C)=\psi\big(\boldsymbol{\xi}(\mathbb C)\big)$.

Centering $\psi_{\mathrm{phys}}$ and $\mathbf{r}$ at the natural state 
guarantees \eqref{eq:natural_state_stress} and avoids additional constraints on network parameters.

% (заменено на TikZ-схему в рис.~\ref{fig:clann_arc})

\textbf{Analytical expressions for energy derivatives}

\textbf{Gradient of the strain energy}

Analytical differentiation of the energy with respect to \(\xi\) yields the gradient:

\begin{equation}
 \vect{r} = \nabla_\xi \psi_{\mathrm{phys}}
   = \vect W_1^T \left( \vect W_2 \odot \Big( 2\,\varphi_\beta(\vect s) \odot \sigma(\beta \vect s) \Big) \right) - \vect r_0,
\label{eq:energy_gradient}
\end{equation}
where $\vect s = \vect W_1 \vect \xi + \vect b_1$, $\sigma(x) = \frac{1}{1 + e^{-x}}$ is the sigmoid, 
and $\odot$ denotes the elementwise (Hadamard) product. 
Compared to the case with $z=\varphi_\beta(s)$, an additional factor $2\,\varphi_\beta(\vect s)$ appears, 
so that the contribution of each hidden neuron to the gradient is scaled by the activation magnitude.

\textbf{Hessian of the strain energy}

Second derivatives of the energy with respect to \(\xi\) define the Hessian, which has the following analytical form:

\begin{equation}
 H_{ij} = \sum_h \eta_h\,W_{2,h}\,W_{1,h i}W_{1,h j},
\label{eq:energy_hessian}
\end{equation}

where, for each hidden unit $h$,
\begin{equation}
  \eta_h
  = 2\Big(\sigma_h^2 + \varphi_\beta(s_h)\,\beta\,\sigma_h(1-\sigma_h)\Big),
\end{equation}
and we use the shorthand $\sigma_h = \sigma(\beta s_h)$, $s_h = (\vect W_1\vect \xi + \vect b_1)_h$.
Here $\sigma' = \beta\,\sigma(1-\sigma)$ is the derivative of the sigmoid. 
The coefficients $\eta_h\ge 0$ together with $W_{2,h}\ge 0$ ensure positive semidefiniteness of the Hessian.

\textbf{Material stability and positive definiteness}

Strict convexity of \(\psi(\xi)\) implies positive definiteness of the Hessian:
\begin{equation}
 \vect H = \frac{\partial^2\psi}{\partial\xi^2} > 0,
\label{eq:positive_hessian}
\end{equation}
This property is important for numerical stability in finite element computations, 
as it improves Newton convergence in practice and prevents singularities in the stiffness matrix.

\begin{figure}[H]
  \centering
  \resizebox{\textwidth}{!}{% ===== CLaNN: однослойная ICNN (наглядная архитектура) =====
\begin{tikzpicture}[
  font=\small, node distance=9mm and 14mm, >={Latex},
  box/.style={draw, rounded corners, align=center, minimum height=7mm, inner sep=2mm},
  % цветовые стили
  block/.style={box, fill=green!6, text width=44mm},
  pin/.style={circle, draw, minimum size=6mm, fill=blue!8},
  neur0/.style={circle, draw, minimum size=6mm, fill=orange!20},
  neur/.style={circle, draw, minimum size=6mm, fill=orange!20},
  cblk/.style={box, fill=gray!15, text width=8mm},
  actblk/.style={box, fill=orange!10, text width=60mm, inner sep=3mm},
  readblk/.style={box, fill=red!8, text width=50mm},
  gradblk/.style={box, fill=purple!10, text width=50mm},
  title/.style={font=\bfseries},
  every fit/.style={draw, rounded corners, inner sep=3mm}
]


% Входы (лог‑лапласовы координаты)
\node[pin] (x1) {$\xi_1$};
\node[pin, below=6mm of x1] (x2) {$\xi_2$};
\node[pin, below=6mm of x2] (x3) {$\xi_3$};
% Блок правее Cauchy–Green тензора C
\node[cblk, left=14mm of x2] (C) {$\mathbb{C}$};
\draw[->] (C.east) -- (x1.west);
\draw[->] (C.east) -- (x2.west);
\draw[->] (C.east) -- (x3.west);

% Аффинная проекция
\node[neur0, right=14mm of x1, yshift=12mm] (h11) {\scriptsize 1};
\node[neur0, right=14mm of x1] (h12) {\scriptsize 2};
\node[neur0, right=14mm of x1, yshift=-12mm] (h13) {\scriptsize 3};
\node[align=center, right=14mm of x1, yshift=-22mm] (vdots0) {$\vdots$};
\node[neur0, right=14mm of x1, yshift=-34mm] (h14) {\scriptsize h};
\draw[->] (x1) -- (h11.west);
\draw[->] (x2) -- (h11.west);
\draw[->] (x3) -- (h11.west);
\draw[->] (x1) -- (h12.west);
\draw[->] (x2) -- (h12.west);
\draw[->] (x3) -- (h12.west);
\draw[->] (x1) -- (h13.west);
\draw[->] (x2) -- (h13.west);
\draw[->] (x3) -- (h13.west);
\draw[->] (x1) -- (h14.west);
\draw[->] (x2) -- (h14.west);
\draw[->] (x3) -- (h14.west);
% \node[block, above=2mm of h11, text width=36mm] (aff) {$s = W_1\,\xi + b_1$};
\node[actblk, above=2mm of h11] (soft) {$z_i = \mathrm{softplus^2}(\beta (W_1\,\xi + b_1))/\beta^2$};

% % Скрытый слой (softplus): три нейрона, троеточие и ещё один
% \node[neur, right=14mm of h12, yshift=12mm] (h1) {};
% \node[neur, right=14mm of h12] (h2) {};
% \node[neur, right=14mm of h12, yshift=-12mm] (h3) {};
% \node[align=center, right=14mm of h12, yshift=-22mm] (vdots1) {$\vdots$};
% \node[neur, right=14mm of h12, yshift=-34mm] (h4) {};
% \draw[->] (h11.east) -- (h1.west);
% \draw[->] (h12.east) -- (h2.west);
% \draw[->] (h13.east) -- (h3.west);
% \draw[->] (h14.east) -- (h4.west);
% % Блок активации над нейронами
% \node[block, above=2mm of h1, text width=36mm] (soft) {$z_i = \mathrm{softplus}(\beta (W_1\,\xi + b_1)/\beta$};

% Линейный выход (неотрицательные веса), центрирован относительно столбца нейронов
\node[readblk, right=22mm of h12, yshift=-11mm] (readout) {$\psi(\vect{\xi}) = \vect{W}_2^{\top}\big(z - z_0\big) - \vect{r}_0^{\top}\vect{\xi}$};
\draw[->] (h11.east) -- (readout.west);
\draw[->] (h12.east) -- (readout.west);
\draw[->] (h13.east) -- (readout.west);
\draw[->] (h14.east) -- (readout.west);

% Производная: второй тензор Пиолы–Кирхгофа
\node[gradblk, below=8mm of readout] (Sblock) {$\mathbb{S} = \frac{\partial \psi}{\partial \vect{\xi}}\, \frac{\partial \vect{\xi}}{\partial \mathbb{C}}$};
\draw[->] (readout.south) -- (Sblock.north);

% % Центрирование \to физическая энергия
% \node[block, below=10mm of readout] (psiphys) {$\psi_{\mathrm{phys}}(\boldsymbol{\xi}) = \psi(\boldsymbol{\xi}) - \mathbf{r}_0^{\top}\boldsymbol{\xi}$ \\
%   \scriptsize $r_0 = W_1^{\top}\!(w_2^{+}\!\odot\!\sigma(\beta b_1))$};
% \draw[->] (readout.south) -- (psiphys.north);

% Рамка и заголовок
\node[fit=(C)(h14)(soft)(readout)(Sblock), label={[title]above:{CLaNN architecture: one-layer ICNN}}] (archfit) {};

\end{tikzpicture}}
  \caption{Schematic of the CLaNN architecture.}
  \label{fig:clann_arc}
  \label{fig:clann_icnn1_nn}
\end{figure}

\begin{figure}[H]
  \centering
  \resizebox{\textwidth}{!}{% ======= CLaNN: FE stretch -> dataset -> CLaNN arch (vertical) -> S -> Train -> (g,H) -> FE inflation =======
% Требует: \usepackage{tikz} \usetikzlibrary{arrows.meta,positioning,fit}
\begin{tikzpicture}[
  font=\small, node distance=8mm and 14mm, >={Latex},
  box/.style={draw, rounded corners, align=center, minimum height=8mm, inner sep=2mm},
  thinbox/.style={draw, rounded corners, align=center, minimum height=6mm, inner sep=1.2mm, text width=52mm},
  inpin/.style={circle, draw, minimum size=5.2mm},
  hid/.style={circle, draw, minimum size=6mm},
  title/.style={font=\bfseries},
  dashedarrow/.style={->, dashed},
  boxL/.style={box, fill=blue!6},
  boxM/.style={box, fill=green!6},
  boxR/.style={box, fill=orange!10},
  trainbox/.style={box, fill=purple!8}
]

% ---------------- Simplified pipeline (3 columns, orthogonal arrows) ----------------
% LEFT COLUMN
\node[boxL, minimum width=46mm, minimum height=32mm] (fe) {FE stretching\\ \scriptsize protocols $p$\\[2pt]
  \includegraphics[width=0.3\linewidth]{../img/Numerical/malt_stress.png}};
\node[boxL, below=8mm of fe, minimum width=46mm] (data) {Data collection\\ $D(p,w)=\{(\mathbb{C},\mathbb{S})\}$};

% MIDDLE COLUMN
\node[boxM, right=28mm of fe, minimum width=54mm] (xi) {Laplace strain\\ $\boldsymbol{\xi}=\boldsymbol{\xi}(\mathbb{C})$};
\node[boxM, below=8mm of xi, minimum width=54mm] (arch) {CLaNN architecture\\ $\psi_{\rm phys}(\boldsymbol{\xi})$};
\node[boxM, below=8mm of arch, minimum width=54mm] (smap) {Autodiff \\[1pt] $\displaystyle \hat{\mathbb S}=\frac{\partial\psi_{\rm phys}}{\partial \mathbb{C}}$};
\node[trainbox, below=8mm of smap, minimum width=54mm] (train) {Training\\ $L=\|\hat{\mathbb S}-\mathbb S\|_{L2}\xrightarrow{\text{Adam}}0$};

% RIGHT COLUMN
\node[boxR, right=28mm of arch, minimum width=50mm] (gh) {Derivatives\\ $g(\boldsymbol{\xi}),\ H(\boldsymbol{\xi})$};
\node[boxR, below=8mm of gh, minimum width=50mm, minimum height=36mm] (infl) {FE inflation tests \\ homogeneous / heterogeneous thickness \\[2pt]
  \includegraphics[width=0.3\linewidth]{../img/Numerical/het/plane.png}};

% ORTHOGONAL ARROWS
\draw[->] (fe) -- (data);
% вилка от датасета: одна ветвь к деформации, другая — к обучению
\path (data.east) -- ++(10mm,0) coordinate (datasplit);
\draw[->] (data.east) -- (datasplit) |- (xi.west);
\draw[->] (datasplit) |- (train.west);
\draw[->] (xi) -- (arch);
\draw[->] (arch) -- (smap);
\draw[->] (smap) -- (train);
\draw[->] (arch.east) -- (gh.west);
\draw[->] (gh.south) -- (infl.north);
% пунктир: обновление параметров из обучения обратно в архитектуру (слева)
\draw[dashedarrow] (train.west) -- ++(-8mm,0) |- (arch.west);

\end{tikzpicture}
}
  \caption{Schematic of the CLaNN computational loop.}
  \label{fig:clann_pipeline}
\end{figure}

This CLaNN architecture ensures all necessary physical properties of the hyperelastic model: 
\textbf{thermodynamic consistency} is achieved through strict compliance with \eqref{eq:chain-rule}, 
which guarantees conservative stresses $\oint \mathbb{S}:\mathrm{d}\mathbb{C} = 0$ and consistency with the laws of 
thermodynamics; 
\textbf{Stress objectivity} holds automatically thanks to the parametrization via the 
Cauchy--Green tensor $\mathbb{C} = \mathbb{F}^{\top}\mathbb{F}$, ensuring invariance with respect to rotations and stress symmetry; 
\textbf{strict non-negativity and coercivity of the energy} are provided by the architectural calibration 
$\psi_{\mathrm{phys}}(\boldsymbol{\xi}) = \mathbf{W}_2^{\top}(z - z_0) - \mathbf{r}_0^{\top}\boldsymbol{\xi}$,
yielding $\psi_{\mathrm{phys}}(\mathbf{0})=0$, $\psi_{\mathrm{phys}}(\boldsymbol{\xi})>0$ for $\boldsymbol{\xi}\ne\mathbf{0}$ and 
$\psi_{\mathrm{phys}}(\boldsymbol{\xi})\to\infty$ as $\|\boldsymbol{\xi}\|\to\infty$; 
finally, the \textbf{physical constraints} \eqref{eq:energy_constraints} are satisfied by the CLaNN network design: monotone, convex activations,
nonnegative output weights, and centering of the strain energy $\psi$ and the response $\vect r$.


