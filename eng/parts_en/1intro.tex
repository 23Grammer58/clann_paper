\section{Introduction}

Mathematical models capable of predicting the nonlinear mechanical behavior of soft materials under large deformations are required across a wide range of engineering fields  --  from the polymer industry to robotics and personalized medicine \cite{shahzad2015mechanical,hauseux2018quantifying,caasenbrood2023control}. The foundation of such models is the nonlinear theory of elasticity \cite{martins2006comparative}, in which the dependence of the stress tensor on variables characterizing the material kinematics is described by constitutive relations (equations of state) \cite{holzapfel2000nonlinear}. In modeling the stress-stretch response of polymers and biological tissues, hyperelastic constitutive laws are widely used \cite{khaniki2023hyperelastic}. In the hyperelastic setting, one postulates the existence of a stretch‑energy density function (potential) $\psi$, depending on a chosen stretch measure, that fully specifies the material's mechanical behavior. The potential must satisfy several requirements: it must reflect material symmetry, be frame‑indifferent (objective), and possess polyconvexity properties \cite{ball1976convexity}, which are sufficient for the existence of solutions to boundary‑value problems in hyperelasticity \cite{ciarlet2021mathematical}.

A large number of hyperelastic models for soft materials have been proposed \cite{chagnon2015hyperelastic}, most of which meet the requirements of material symmetry, objectivity, and polyconvexity by employing an invariant‑based approach. That is, for a chosen stretch measure one specifies a set of invariants, and the stretch‑energy potential is taken as a function of these invariants. A common practice is to use invariants of the right/left Cauchy-Green deformation tensors. For isotropic materials the stretch‑energy potential can be expressed as a function of the three invariants of the right Cauchy-Green deformation tensor,
$\psi=\psi_{\mathrm{vol}}(J)+\psi_{\mathrm{iso}}(I_1,I_2)$,
where $J$ is the Jacobian (determinant of the deformation gradient) representing the local volume change, and $I_1, I_2, I_3$ are the invariants of the right Cauchy-Green tensor. A further extension of the invariant approach introduces the so‑called pseudo‑invariants of the right Cauchy-Green tensor, $I_4,\ldots,I_8$, enabling the description of transversely isotropic and orthotropic material classes \cite{holzapfel2000nonlinear}.

This approach requires prescribing the stretch‑energy potential a priori as an analytic function with parameters identified from experimental data. Its main shortcomings are the non‑uniqueness of the optimal parameter set, the lack of a direct physical interpretation of the invariants in terms of stretch \cite{srinivasa2012use} -- which imposes stringent demands on experimental testing, namely the achievement of homogeneous stretch and stress fields -- and the subjective choice of the potential's form from among many expert‑constructed models \cite{xi2023}.

To some extent these drawbacks are alleviated by constructing an optimal hyperelastic model via regression over a preselected dictionary of invariant‑based monomials \cite{linka2023new}, or by reducing generalized models using information‑theoretic analysis of experimental data \cite{guan2019aic}. In combination with full‑field experimental stretch measurement methods (digital image correlation, DIC \cite{beberniss2017high}), the Virtual Fields Method (VFM) \cite{pierron2012virtual}, and inverse finite‑element (inverse FE) approaches \cite{nguyen2011inverse}, this becomes a powerful toolkit for modeling material mechanics within hyperelasticity. Nevertheless, such approaches remain phenomenological and still require expert model selection.

It is not necessary to know the analytic form of the stretch‑energy potential to build a hyperelastic model, which is an advantage that enables data‑driven modeling. For a hyperelastic material, specifying the constitutive law requires only the derivatives of the stretch‑energy potential with respect to the chosen stretch measure -- the so‑called response functions \cite{holzapfel2000nonlinear}. Using full‑field methods for experimental stretch and stress evaluation (DIC and related techniques) \cite{romo2014vitro}, the response functions can be constructed directly from experimental data obtained over a wide range of deformation modes. This motivates the development of data‑driven approaches to hyperelastic modeling \cite{KirchdoerferOrtiz2016}.

In \cite{xi2023}, a data‑driven method for direct modeling of the mechanics of isotropic and anisotropic materials is proposed. It employs response functions based on a physically interpretable Laplace stretch measure \cite{freed2020laplace}, thereby bypassing the issues of an invariant formulation by constructing the response functions directly from experimental data. No prior knowledge of material symmetry is required. The collection of response functions forms a tabulated constitutive relation. Direct finite element modeling was performed using the hyperelastic nodal force method and k-NN interpolation of the response in a three-dimensional space of Laplace stretch measures, required for the iterative equilibrium calculation. The nonlinear algebraic systems arising in virtual quasi‑static extension and inflation problems are solved by a simple relaxation scheme, in which inverse‑distance‑weighting interpolation is used at each iteration to evaluate the response functions at any point in the Laplace‑stretch space. The limitations of this approach are the need for sufficiently "rich" data and the inability to apply gradient‑based solvers to the nonlinear algebraic systems due to the non‑smoothness of the interpolation of the tabulated constitutive relation and the lack of guarantees of energy convexity.

In parallel, physics‑informed neural approaches are being developed. In particular, by using input‑convex neural networks (ICNNs) \cite{icnn2017} and enforcing monotone non‑decrease of the stretch‑energy with respect to the invariants, one can guarantee polyconvexity \cite{BoydVandenberghe2004}, thereby satisfying the requirements for hyperelastic potentials \cite{tacc2024benchmarking}. The work \cite{kalina2025neural} presents an invariant architecture of a physics‑informed neural network compatible with finite‑element packages. Despite interpretability and thermodynamic consistency, the network architecture incorporates a set of assumptions -- generalized structure tensors \cite{schroder2008anisotropic} -- which effectively fixes the material symmetry class.

In this work we propose an approach that combines the advantages of a tabulated constitutive representation in Laplace stretch measures \cite{ddaniso2024} with physics‑augmented neural networks based on ICNNs \cite{icnn2017}, satisfying the requirements of hyperelastic material models. We formulate a hyperelastic potential that is thermodynamically consistent, objective by construction, convex in its inputs, and does not require prior knowledge of material symmetry. The smoothness of the approximation ensures compatibility with gradient‑based solvers for nonlinear algebraic systems.

\paragraph{Scope and limitations}
quasi-statics, membrane formulation,

\paragraph{Organization of the paper}


