\section{Stress and thermodynamic consistency}

We use the second Piola--Kirchhoff stress as the stress measure and compute it by differentiating the energy $\psi$ via the chain rule with respect to the right Cauchy--Green tensor $\mathbb C$:
\begin{equation}
  \mathbb S \,=\, 2\,\frac{\partial \psi}{\partial \mathbb C}
  \,=\, 2\,\frac{\partial \psi}{\partial \vect{\xi}} : \frac{\partial \vect{\xi}}{\partial \mathbb C}
  \,=\, 2\,\vect r(\vect{\xi}) : \frac{\partial \vect{\xi}}{\partial \mathbb C},\qquad \vect r := \frac{\partial \psi}{\partial \vect{\xi}}.
  \label{eq:chain-rule-en}
\end{equation}
This yields objectivity, symmetry of $\mathbb S$, and compliance with the Clausius--Duhem inequality in the hyperelastic (non-dissipative) setting. In 2D, the components read
\begin{equation}
\begin{aligned}
  S_{11} &= e^{-2\xi_1}\big(r_1-2\xi_3 r_3\big) + e^{-2\xi_2} \, r_2\,\xi_3^2,\\
  S_{22} &= e^{-2\xi_2} \, r_2,\\
  S_{12} &= -e^{-2\xi_2} \, r_2\,\xi_3 + e^{-2\xi_1} \, r_3.
\end{aligned}
\label{eq:stress_components_2d-en}
\end{equation}

The hyperelastic potential must satisfy: (i) non-negativity $\psi(\vect{\xi})\ge 0$; (ii) zero energy and stress in the undeformed state, $\psi(\vect 0)=0$, $\mathbb S(\mathbb I)=\mathbb 0$; and (iii) coercivity/growth for large deformations.

% (Russian duplicate section removed)
The vector $\frac{\partial \boldsymbol\xi}{\partial \mathbf{C}}$ is called the basis; it is a known analytical expression depending on the chosen strain measure; $\vect{r} = (r_1, r_2, r_3)$ is the response function, which is the ultimate \textit{quantity sought} in training the \textit{data-driven} constitutive relation.  

This construction has key consequences:
\newline
\textit{Objectivity:} $\psi(\mathbb{C})=\psi(\mathbb{Q}^\top\mathbb{C}\mathbb{Q})$ for any orthogonal $\mathbb{Q}$; hence $\mathbb{S}$ is invariant under rotations.
\newline
\textit{Stress symmetry:} $\mathbb{S}=\mathbb{S}^\top$ due to the symmetry of $\mathbb{C}$ and proper application of the chain rule.
\newline
\textit{Thermodynamic consistency:} identity \eqref{eq:chain-rule-en} follows from the Clausius--Duhem inequality
$\mathcal{D} = \mathbb{S} : \dot{\mathbb{C}} - \dot{\psi}(\mathbb{C}) \geq 0$,
which expresses the second law of thermodynamics for mechanical processes \cite{truesdell1984historical,truesdell2004nonlinear}.
% \begin{itemize}
%   \item \textbf{Объективность:} $\psi(\mathbf{C})=\psi(\mathbf{Q}^\top\mathbf{C}\mathbf{Q})$ для любой ортогональной $\mathbf{Q}$, а значит и $\mathbf{S}$ инвариантен к поворотам.
%   \item \textbf{Симметрия напряжений:} $\mathbf{S}=\mathbf{S}^\top$ вследствие симметрии $\mathbf{C}$ и корректного применения цепного правила.
%   \item \textbf{Термодинамическая корректность:} равенство \eqref{eq:chain-rule} является следствием неравенства Клаузиуса-Дюгема 
%   $\mathcal{D} = \mathbf{S} : \dot{\mathbf{C}} - \dot{\psi}(\mathbf{C}) \geq 0$, 
%   выражающее второе начало термодинамики для механических процессов \cite{truesdell1984historical,truesdell2004nonlinear}.
% \end{itemize}

% \textbf{Связь тензора Лапласа и второго тензора напряжений Пиолы-Кирхгофа}

 % Необходимо установить связь деформации Лапласа $\vect{\xie}$ и второго тензора напряжений Пиолы-Кирхгофа $\vect{S}$. Применяя цепное правило дифференцирования к выражению \eqref{eq:chain-rule} и используя меру деформации Лапласа,

% Link between Laplace strain tensor and the second Piola–Kirchhoff stress tensor

% Establishing the relation using the chain rule with the Laplace strain measure
Writing equation \eqref{eq:chain-rule} componentwise and substituting the known basis $\frac{\partial \vect{\xi}}{\partial \mathbb{C}}$ yields analytical expressions for the components of the second Piola--Kirchhoff stress in the two-dimensional case:

\begin{equation}
\begin{aligned}
  S_{11} &= e^{-2\xi_1}\big(r_1-2\xi_3 r_3\big) + e^{-2\xi_2} r_2\,\xi_3^2,\\
  S_{22} &= e^{-2\xi_2} r_2,\\
  S_{12} &= -e^{-2\xi_2} r_2\,\xi_3 + e^{-2\xi_1} r_3,
\end{aligned}
\label{eq:stress_components_2d}
\end{equation}
% r1, r2, r3 are components of the response function r = d psi / d xi

\textbf{Constraints on the hyperelastic model}

In accordance with the principles of thermodynamics and continuum mechanics,
the hyperelastic model $\psi$ must satisfy a set of fundamental constraints ensuring physical correctness and
material stability:
\newline
\textit{Non-negativity} excludes negative strain energy
  \begin{equation}
    \psi(\vect{\xi}) \ge 0\quad \forall\,\vect{\xi}\in\mathbb{R}^3.
    \label{eq:non_negativity}
  \end{equation}
  % (omitted)
\newline  
 \textit{Zero values for $\psi$ and $\vect S$ in the natural state} mean that the undeformed (initial) configuration has no residual stresses
  \begin{equation}
    \psi(\vect 0)=0,\qquad \vect S(\vect I)=\vect 0,
    \label{eq:natural_state_stress}
  \end{equation}
\newline
  \textit{Unbounded growth (coercivity)}. Unbounded growth of the strain measure is physically unattainable at finite work
  \begin{equation}
    \psi(\vect{\xi}) \to \infty\ \text{as}\ \lVert\vect{\xi}\rVert\to\infty,\qquad
    \mathbb{S} \to \infty\ \text{as}\ J\to\infty \text{ or } J\to 0^{+},\qquad
    J=\det\vect F,
    \label{eq:energy_constraints}
  \end{equation}
  % Coercivity remark (omitted in English build)

Properties \eqref{eq:non_negativity}--\eqref{eq:energy_constraints} are commonly written via the deformation gradient \(\vect{F}\) and the right Cauchy--Green tensor \(\vect{C}\)
\cite{antman2005nonlin,green1839laws,kirchhoff1850gleichgewicht}, but they are equivalent for the Laplace strain measure \(\vect{\xi}\) as well.

% Properties \eqref{eq:non_negativity}--\eqref{eq:energy_constraints}


